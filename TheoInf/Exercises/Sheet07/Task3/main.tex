\documentclass[14pt]{article}

% Packages
\usepackage[utf8]{inputenc}
\usepackage[german]{babel}
\usepackage{amssymb}
\usepackage{fancyhdr}
\usepackage[framemethod=TikZ]{mdframed}
\usepackage{amsthm}
\usepackage{amsmath}
\usepackage[T1]{fontenc}
\usepackage{mathabx}
\usepackage{listings}

\title{Theoretische Informatik}
\author{Julian Schubert}

% Definition
\mdfdefinestyle{theoremstyle}{
    linecolor=blue!20,
    linewidth=2pt,
    frametitlerule=true,
    frametitlebackgroundcolor=gray!20,
    innertopmargin=\topskip
}
\mdtheorem[style=theoremstyle]{definition}{Definition}

% Eigenschaft
\mdfdefinestyle{eigenschaftstyle}{
    linecolor=red!50,
    linewidth=2pt,
    frametitlerule=true,
    frametitlebackgroundcolor=gray!20,
    innertopmargin=\topskip
}
\mdtheorem[style=eigenschaftstyle]{eigenschaft}{Eigenschaft}


% Kopf- / Fußzeile
\makeatletter
\let\runauthor\@author
\let\runtitle\@title
\pagestyle{fancy}
\fancyhf{}
\rhead{\runtitle}
\lhead{\runauthor}
\cfoot{\thepage}

\newcommand{\subsubsubsection}[1]{\paragraph{#1}\mbox{}\\}
\setcounter{secnumdepth}{4}
\setcounter{tocdepth}{4}


\begin{document}
\section*{Aufgabe 3}
\subsection*{a}
Wenn eine Wort $w$ vom Automaten A akzeptiert wird, dann befindet sich der
Automat am ende in einem akzeptierendem Zustand $z_k \in F$. Wenn wir das 
Wort $w^R$ akzeptieren möchten, so müssen wir in diesem akzeptierten 
Zustand $z_k$ starten und dann alle Pfeile umdrehen. Sprich, wenn man 
im Automaten $A$ vom Zustand $z_i$ bei Eingabe $e$ in $z_j$ landet, so
muss man in unserem Spiegelautomaten vom Zustand $z_j$ bei Eingabe $e$
in $z_i$ landen. Unser neuer Spiegelautomat hat nur einen akzeptierenden
Zustand, und zwar $z_0$ des Uhrsprünglichen Automatens da wenn wir im
Spiegelautomaten von einem akzeptierenden Zustand starten und wieder bei
$z_0$ von $A$ landen haben wir ein Wort aus $L(A)$ rückwährts eingelesen.
Zuletzt muss beachtet werden, dass $A$ mehrere akzeptierte Zustände haben kann,
unser neuer Automat $N$ müsste also mehrere Startzustände haben, was per
Definition nicht erlaubt ist. Wir müssen also einen neuen Startzustand $z_0'$
einführen der alle akzeptieren Zustände aus A zusammenfasst, die Pfeile 
von diesem Zustand weg sind wie oben beschrieben die aus A umgedrehten
Übergänge. Die alten akzeptierenden Zustände aus A müssen in N entfernt
werden. \\
Die Überführungsfunktion von N muss also so aufgestellt werden, dass von einem 
Zustand $z$ bei Eingabe $a$ in alle Zustände $z_i$ übergegangen wird, für 
die der Automat A von $z_i$ bei Eingabe $a$ nach $z$ wechselt.
\subsection*{b}
DEA $N = (\varSigma, Z, \delta', z_k, F')$ mit
\begin{itemize}
    \item $\varSigma_N = \varSigma_A$, $Z_N = \{Z_A \cup \{z_0'\}\} \backslash F_A$, $F' = \{ z_0 \}$
    \item $\delta'(z, a)) = \bigcup_{z_i \in Z_N} \{ z_i | \delta(z_i, a) = z \}$
\end{itemize}
\end{document}