\documentclass[14pt]{article}

% Packages
\usepackage[utf8]{inputenc}
\usepackage[german]{babel}
\usepackage{amssymb}
\usepackage{fancyhdr}
\usepackage[framemethod=TikZ]{mdframed}
\usepackage{amsthm}
\usepackage{amsmath}
\usepackage[T1]{fontenc}
\usepackage{mathabx}
\usepackage{listings}

\title{Theoretische Informatik}
\author{Julian Schubert}

% Definition
\mdfdefinestyle{theoremstyle}{
    linecolor=blue!20,
    linewidth=2pt,
    frametitlerule=true,
    frametitlebackgroundcolor=gray!20,
    innertopmargin=\topskip
}
\mdtheorem[style=theoremstyle]{definition}{Definition}

% Eigenschaft
\mdfdefinestyle{eigenschaftstyle}{
    linecolor=red!50,
    linewidth=2pt,
    frametitlerule=true,
    frametitlebackgroundcolor=gray!20,
    innertopmargin=\topskip
}
\mdtheorem[style=eigenschaftstyle]{eigenschaft}{Eigenschaft}


% Kopf- / Fußzeile
\makeatletter
\let\runauthor\@author
\let\runtitle\@title
\pagestyle{fancy}
\fancyhf{}
\rhead{\runtitle}
\lhead{\runauthor}
\cfoot{\thepage}

\newcommand{\subsubsubsection}[1]{\paragraph{#1}\mbox{}\\}
\setcounter{secnumdepth}{4}
\setcounter{tocdepth}{4}


\begin{document}
\section*{Aufgabe 4}
Da für jede Spraceh L $R(R(L)) = L$ gilt, können wir einen NEA konstruieren,
der nur einen akzeptierenden Zustand besitzt. Dies funktioniert, indem wir 
Den Automaten für R(L) wie in Aufgabe 3 konstruieren, und dann für diesen
neuen Automaten nochmal R(R(L)) konstruieren. Der Automat der dann 
R(R(L)) akzeptiert enthält (wie in Aufgabe 3 gezeigt) nur einen einzigen 
akzeptierenden Zustand und da $R(R(L)) = L$ gilt, akzeptiert dieser 
auch die richtige Sprache
\end{document}