\documentclass[14pt]{article}

% Packages
\usepackage[utf8]{inputenc}
\usepackage[german]{babel}
\usepackage{amssymb}
\usepackage{fancyhdr}
\usepackage[framemethod=TikZ]{mdframed}
\usepackage{amsthm}
\usepackage{amsmath}
\usepackage[T1]{fontenc}
\usepackage{mathabx}

\title{Theoretische Informatik}
\author{Julian Schubert}

% Definition
\mdfdefinestyle{theoremstyle}{
    linecolor=blue!20,
    linewidth=2pt,
    frametitlerule=true,
    frametitlebackgroundcolor=gray!20,
    innertopmargin=\topskip
}
\mdtheorem[style=theoremstyle]{definition}{Definition}

% Eigenschaft
\mdfdefinestyle{eigenschaftstyle}{
    linecolor=red!50,
    linewidth=2pt,
    frametitlerule=true,
    frametitlebackgroundcolor=gray!20,
    innertopmargin=\topskip
}
\mdtheorem[style=eigenschaftstyle]{eigenschaft}{Eigenschaft}


% Kopf- / Fußzeile
\makeatletter
\let\runauthor\@author
\let\runtitle\@title
\pagestyle{fancy}
\fancyhf{}
\rhead{\runtitle}
\lhead{\runauthor}
\cfoot{\thepage}

\newcommand{\subsubsubsection}[1]{\paragraph{#1}\mbox{}\\}
\setcounter{secnumdepth}{4}
\setcounter{tocdepth}{4}


\begin{document}


\maketitle

\section*{Aufgabe 1}
S ist Menge aller ber3echen n -> n mit der eigenschaft (dem was in B steht).
\subsection*{a}
Wir können ein Programm angeben, das die characteristische Funktion $\chi_A$ berechnet. Damit
ist die Menge A entscheidbar.
\subsection*{b}
Wir konstruieren zunächst eine Menge S für die gilt: \\
\begin{center}
    $I(S) = \{ i \in \mathbb{N} | \text{ die von $M_i$ berechnete Funktion erfüllt die
    Eigenschaft } S\}$ \\
    $S = B$
\end{center}
Wir zeigen zunächst $S \neq \emptyset$:
\begin{center}
    $f(x) = x \in S$
\end{center}
Nun zeigen wir, dass $S$ eine echte Teilmege aller berechenbaren Funktionen $\mathbb{N}^n
\rightarrow \mathbb{N}$ ist:
\begin{center}
    $g(x) = n.d \notin S$
\end{center}
Mit dem Satzt von Rice folgt hiermit die unentscheidbarkeit von $B$
\subsection*{c}
Wir konstruieren erneut die Menge $S$. Wir zeigen $S \neq \emptyset$:
\begin{center}
    $f(x) = n.d \in S$
\end{center}
$S$ ist eine echte Teilmenge aller berechenbaren Funktionen $\mathbb{N}^n
\rightarrow \mathbb{N}$:
\begin{center}
    $g(x) = 3 \notin S$
\end{center}
Mit dem Satzt von Rice folgt hiermit die unentscheidbarkeit von $C$
\end{document}