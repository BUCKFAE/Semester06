\documentclass[14pt]{article}

% Packages
\usepackage[utf8]{inputenc}
\usepackage[german]{babel}
\usepackage{amssymb}
\usepackage{fancyhdr}
\usepackage[framemethod=TikZ]{mdframed}
\usepackage{amsthm}
\usepackage{amsmath}
\usepackage[T1]{fontenc}
\usepackage{mathabx}
\usepackage{listings}

\title{Theoretische Informatik}
\author{Julian Schubert}

% Definition
\mdfdefinestyle{theoremstyle}{
    linecolor=blue!20,
    linewidth=2pt,
    frametitlerule=true,
    frametitlebackgroundcolor=gray!20,
    innertopmargin=\topskip
}
\mdtheorem[style=theoremstyle]{definition}{Definition}

% Eigenschaft
\mdfdefinestyle{eigenschaftstyle}{
    linecolor=red!50,
    linewidth=2pt,
    frametitlerule=true,
    frametitlebackgroundcolor=gray!20,
    innertopmargin=\topskip
}
\mdtheorem[style=eigenschaftstyle]{eigenschaft}{Eigenschaft}


% Kopf- / Fußzeile
\makeatletter
\let\runauthor\@author
\let\runtitle\@title
\pagestyle{fancy}
\fancyhf{}
\rhead{\runtitle}
\lhead{\runauthor}
\cfoot{\thepage}

\newcommand{\subsubsubsection}[1]{\paragraph{#1}\mbox{}\\}
\setcounter{secnumdepth}{4}
\setcounter{tocdepth}{4}


\begin{document}
\section*{Aufgabe 1}
Es gilt 
\begin{equation*}
    L = \{ a^m | m \text{ ist eine Quadratzahl} \} = \{ a^{n \cdot n} \}, 
    n \in \mathbb{N} 
\end{equation*}

\begin{itemize}
    \item Sei $n$ eine beliebige Zahl aus $\mathbb{N}$ mit $n \geq 10$
    \item Wir definieren $w = a^n \cdot a^n$ \\
    (Damit liegt $w \in L$ und $|w| \geq n$)
    \item Sei $w = xyz$ eine beliebige Zerlegung mit $y \neq \epsilon$ und
    $|xy| \leq n$
    \item Wir definieren $i = 0$ und zeigen $xy^iz \notin L$
    \begin{itemize}
        \item Aus $|xy| \leq n$ folgt: \\
        $x = a^j$ und $y = a^k$ für $j, k \geq 0$ \\
        aus $y \neq \epsilon$ folgt $k \geq 1$
    \end{itemize}
    \item Daraus folgt $w = xyz = a^ja^ka^{n - (j+k)}$
\end{itemize}
\end{document}