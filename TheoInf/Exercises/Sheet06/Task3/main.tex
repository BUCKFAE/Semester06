\documentclass[14pt]{article}

% Packages
\usepackage[utf8]{inputenc}
\usepackage[german]{babel}
\usepackage{amssymb}
\usepackage{fancyhdr}
\usepackage[framemethod=TikZ]{mdframed}
\usepackage{amsthm}
\usepackage{amsmath}
\usepackage[T1]{fontenc}
\usepackage{mathabx}
\usepackage{listings}

\title{Theoretische Informatik}
\author{Julian Schubert}

% Definition
\mdfdefinestyle{theoremstyle}{
    linecolor=blue!20,
    linewidth=2pt,
    frametitlerule=true,
    frametitlebackgroundcolor=gray!20,
    innertopmargin=\topskip
}
\mdtheorem[style=theoremstyle]{definition}{Definition}

% Eigenschaft
\mdfdefinestyle{eigenschaftstyle}{
    linecolor=red!50,
    linewidth=2pt,
    frametitlerule=true,
    frametitlebackgroundcolor=gray!20,
    innertopmargin=\topskip
}
\mdtheorem[style=eigenschaftstyle]{eigenschaft}{Eigenschaft}


% Kopf- / Fußzeile
\makeatletter
\let\runauthor\@author
\let\runtitle\@title
\pagestyle{fancy}
\fancyhf{}
\rhead{\runtitle}
\lhead{\runauthor}
\cfoot{\thepage}

\newcommand{\subsubsubsection}[1]{\paragraph{#1}\mbox{}\\}
\setcounter{secnumdepth}{4}
\setcounter{tocdepth}{4}


\begin{document}
\section*{Aufgabe 3}
Wir definieren die funktion h wie folgt:
\begin{equation*}
    h(x) =
    \begin{cases}
        g(\pi_1)                 & \text{ falls $h(\pi_1)$ nach maximal
        $\pi_2$ vielen Rechenschritten ein Ergebnis liefert.}             \\
        0                   & \text{sonst}
    \end{cases}
\end{equation*}
Wir benutzen hierfür die Cantorsche Paarungsfunktion. Wir konstruieren
ein Python-Programm das g(x) berechnet. Stoppt das Programm nicht nach
$\pi_2$ vielen Schritten, so geben wir 0 aus (da g(0)  = 0). Damit ist
der Definitionsbereich userer Funktion gleich dem Definitionsbereich von
g, da er nur Werte enthält auf denen g(0) nach endlich vielen Schritten
hält.
\end{document}