\documentclass[14pt]{article}

% Packages
\usepackage[utf8]{inputenc}
\usepackage[german]{babel}
\usepackage{amssymb}
\usepackage{fancyhdr}
\usepackage[framemethod=TikZ]{mdframed}
\usepackage{amsthm}
\usepackage{amsmath}
\usepackage[T1]{fontenc}
\usepackage{mathabx}
\usepackage{listings}

\title{Theoretische Informatik}
\author{Julian Schubert}

% Definition
\mdfdefinestyle{theoremstyle}{
    linecolor=blue!20,
    linewidth=2pt,
    frametitlerule=true,
    frametitlebackgroundcolor=gray!20,
    innertopmargin=\topskip
}
\mdtheorem[style=theoremstyle]{definition}{Definition}

% Eigenschaft
\mdfdefinestyle{eigenschaftstyle}{
    linecolor=red!50,
    linewidth=2pt,
    frametitlerule=true,
    frametitlebackgroundcolor=gray!20,
    innertopmargin=\topskip
}
\mdtheorem[style=eigenschaftstyle]{eigenschaft}{Eigenschaft}


% Kopf- / Fußzeile
\makeatletter
\let\runauthor\@author
\let\runtitle\@title
\pagestyle{fancy}
\fancyhf{}
\rhead{\runtitle}
\lhead{\runauthor}
\cfoot{\thepage}

\newcommand{\subsubsubsection}[1]{\paragraph{#1}\mbox{}\\}
\setcounter{secnumdepth}{4}
\setcounter{tocdepth}{4}


\begin{document}
\section*{Aufgabe 4}
Wir zeigen zunächst das gilt:
\[
    (x)^2 < (x + 1)^2  \\
    (x)^2 < x^2 + 2x + 1
\]
Daraus folgt: \\
\[
    x^2 - y^2 < (x + 1)^2 - (y + 1)^2  
\]
Wir können also ein x finden, für das gilt:
\[
    (x + 1)^2 - x^2 \geq d
\]
Wenn wir dieses $x = x_{max}$ gefunden haben können wir für alle (endlich viele)
Kombinationen von x und y ausprobieren, die kleiner (gleich) $x_{max}$ sind
und überprüfen, obe eine dieser endlich vielen Kombinationen die Bedingung
erfüllt. \\
$\Rightarrow$ die Menge A liegt damit in REC
\end{document}