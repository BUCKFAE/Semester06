\documentclass[14pt]{article}

% Packages
\usepackage[utf8]{inputenc}
\usepackage[german]{babel}
\usepackage{amssymb}
\usepackage{fancyhdr}
\usepackage[framemethod=TikZ]{mdframed}
\usepackage{amsthm}
\usepackage{amsmath}
\usepackage[T1]{fontenc}
\usepackage{mathabx}
\usepackage{listings}

\title{Theoretische Informatik}
\author{Julian Schubert}

% Definition
\mdfdefinestyle{theoremstyle}{
    linecolor=blue!20,
    linewidth=2pt,
    frametitlerule=true,
    frametitlebackgroundcolor=gray!20,
    innertopmargin=\topskip
}
\mdtheorem[style=theoremstyle]{definition}{Definition}

% Eigenschaft
\mdfdefinestyle{eigenschaftstyle}{
    linecolor=red!50,
    linewidth=2pt,
    frametitlerule=true,
    frametitlebackgroundcolor=gray!20,
    innertopmargin=\topskip
}
\mdtheorem[style=eigenschaftstyle]{eigenschaft}{Eigenschaft}


% Kopf- / Fußzeile
\makeatletter
\let\runauthor\@author
\let\runtitle\@title
\pagestyle{fancy}
\fancyhf{}
\rhead{\runtitle}
\lhead{\runauthor}
\cfoot{\thepage}

\newcommand{\subsubsubsection}[1]{\paragraph{#1}\mbox{}\\}
\setcounter{secnumdepth}{4}
\setcounter{tocdepth}{4}


\begin{document}
\section*{Aufgabe 2}
Wenn A semientscheidbar ist, dann muss A aufzählbar sein.
Die semicharakteristische Funktion ist berechenbar:
\begin{equation*}
    \chi (x) =
    \begin{cases}
        1                 & \text{falls eine Primzahle $p \in \mathbb{P}$
        existiert, mit $f(p) = y$}             \\
        \text{n.d}                   & \text{sonst}
    \end{cases}
\end{equation*}
Dies ist gegeben da wenn eine solche Zahl existiert, wir sie in 
endlicher Zeit finden, sonst laufen wir unendlich lang was die 
semientscheidbarkeit nicht verletzt.
\end{document}