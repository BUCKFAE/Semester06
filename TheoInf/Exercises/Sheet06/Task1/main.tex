\documentclass[14pt]{article}

% Packages
\usepackage[utf8]{inputenc}
\usepackage[german]{babel}
\usepackage{amssymb}
\usepackage{fancyhdr}
\usepackage[framemethod=TikZ]{mdframed}
\usepackage{amsthm}
\usepackage{amsmath}
\usepackage[T1]{fontenc}
\usepackage{mathabx}
\usepackage{listings}

\title{Theoretische Informatik}
\author{Julian Schubert}

% Definition
\mdfdefinestyle{theoremstyle}{
    linecolor=blue!20,
    linewidth=2pt,
    frametitlerule=true,
    frametitlebackgroundcolor=gray!20,
    innertopmargin=\topskip
}
\mdtheorem[style=theoremstyle]{definition}{Definition}

% Eigenschaft
\mdfdefinestyle{eigenschaftstyle}{
    linecolor=red!50,
    linewidth=2pt,
    frametitlerule=true,
    frametitlebackgroundcolor=gray!20,
    innertopmargin=\topskip
}
\mdtheorem[style=eigenschaftstyle]{eigenschaft}{Eigenschaft}


% Kopf- / Fußzeile
\makeatletter
\let\runauthor\@author
\let\runtitle\@title
\pagestyle{fancy}
\fancyhf{}
\rhead{\runtitle}
\lhead{\runauthor}
\cfoot{\thepage}

\newcommand{\subsubsubsection}[1]{\paragraph{#1}\mbox{}\\}
\setcounter{secnumdepth}{4}
\setcounter{tocdepth}{4}


\begin{document}
\section*{Aufgabe 1}
\subsection*{a}
Wir zeigen zunächst das die Menge A nicht leer ist: \\
\[
    3^1 \cdot 5^1 = 15  
\]
$\Rightarrow$ die Zahl 15 liegt in der Menge A, damit ist die
Menge A nicht leer. \\
Allerdings ist A keine echte Teilmene der berechenbaren Funktionen, A
ist also berechenbar. Dies ist gegeben da wir für jedes n ein größtes
mögliches p und ein größtes mögliches q finden können da gelten muss: \\
\[
    p \leq n \land q \leq n  
\]
Wir können also endlich viele p-q-kombinationen durchprobieren. Auch 
für die Potenzen können wir immer eine obere Schranke finden da mit 
erhöhung der Potenzen auch der resultierende Wert von $p^iq^j$ steigt \\
Folgender Algorithmus zeigt die Berechenbarkeit von A
\begin{lstlisting}[language=Python]
def is_prime(x):
    return False if x < 2 else all([x % i != 0 for i in range(2, x - 1)])


def check(n):


    for p in range(2, n):

        # Skipping if p is not a prime
        if not is_prime(p):
            continue

        for q in range(2, n):
            
            # Skipping if q is not a prime
            if not is_prime(q):
                continue

            i = 1  
            j = 1

            # Increasing i and j until the result gets to big
            while (p ** i) * (q **j) <= n:

                # Checking if we found a valid solution
                if (p ** i) * (q ** j) == n:
                    return (p, q, i, j)  

                i += 1

                # Ensuring we test every valid combination of i and j
                if j == i:
                    j += 1
                    i = 1

    return None


# Valdiating results
for i in range(50):
    res = check(i)
    sol = (res[0] ** res[2]) * (res[1] ** res[3]) if not res is None else ""
    print(f"{i}: {res}: {sol}")
\end{lstlisting}
\subsection*{b}
Wir zeigen zunächst das die Menge B nicht leer ist: \\
Die Funktion $f: \mathbb{N} \rightarrow \mathbb{N}$ mit $f(x) = x$ liegt
in $M_i$ und ist entscheidbar. Daher ist B nicht leer. \\
Die Funktion 
\begin{equation*}
    g(x) =
    \begin{cases}
        2                 & \text{falls $x = 3$}             \\
        x                   & \text{sonst}
    \end{cases}
\end{equation*}
ist berechenbar, liegt also in $M_i$, jedoch nicht in B, da $3$ zwar 
in $D_g$ liegt, aber nicht in $W_g$. Dadurch
ist B eine echte Teilmenge der berechenbaren Funktionen $\mathbb{N}^n
\rightarrow \mathbb{N}$ und ist somit laut dem Satz von Rice
unentscheidbar.
\subsection*{c}
Die Menge C ist nicht leer, da die Maschiene $M_k$ mit der Eigenschaft,
dass sie für keine Eingabe hält in $M_i$ und somit in C liegt. \\
Ebenso liegt die Maschiene die einfach die Eingabe wieder ausgibt
($f(x) = x$) in $M_i$, jedoch nicht in C (da diese Maschiene auf
geraden Eingaben hält). \\
Damit ist C laut satz von Rice untentscheidbar.
\subsection*{d}
Eine Maschiene $M_i$ liegt genau dann in i, wenn sie das spezielle
Halteproblem löst. Das spezielle Halteproblem ist nicht entscheidbar,
daher ist die Menge D auch nicht entscheidbar (da für kein Element
in der Menge entschieden werden kann, ob sie das spezielle Halteproblem
löst)
\subsection*{e}
Die Menge E ist keine echte Teilmenge aller berechenbaren Funktionen.
Dies ist gegeben da wenn eine Funktion auf $D = \mathbb{N}^n$ berechenbar 
ist, dann enthält der Definitionsbereich mehr als 2020 Eingaben, die
TM hält also auf mehr als $2020$ Eingaben. Durch den Satz von Rice
ist diese Menge somit entscheidbar.
\end{document}