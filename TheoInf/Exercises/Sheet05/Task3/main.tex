\documentclass[14pt]{article}

% Packages
\usepackage[utf8]{inputenc}
\usepackage[german]{babel}
\usepackage{amssymb}
\usepackage{fancyhdr}
\usepackage[framemethod=TikZ]{mdframed}
\usepackage{amsthm}
\usepackage{amsmath}
\usepackage[T1]{fontenc}
\usepackage{mathabx}
\usepackage{listings}

\title{Theoretische Informatik}
\author{Julian Schubert}

% Definition
\mdfdefinestyle{theoremstyle}{
    linecolor=blue!20,
    linewidth=2pt,
    frametitlerule=true,
    frametitlebackgroundcolor=gray!20,
    innertopmargin=\topskip
}
\mdtheorem[style=theoremstyle]{definition}{Definition}

% Eigenschaft
\mdfdefinestyle{eigenschaftstyle}{
    linecolor=red!50,
    linewidth=2pt,
    frametitlerule=true,
    frametitlebackgroundcolor=gray!20,
    innertopmargin=\topskip
}
\mdtheorem[style=eigenschaftstyle]{eigenschaft}{Eigenschaft}


% Kopf- / Fußzeile
\makeatletter
\let\runauthor\@author
\let\runtitle\@title
\pagestyle{fancy}
\fancyhf{}
\rhead{\runtitle}
\lhead{\runauthor}
\cfoot{\thepage}

\newcommand{\subsubsubsection}[1]{\paragraph{#1}\mbox{}\\}
\setcounter{secnumdepth}{4}
\setcounter{tocdepth}{4}


\begin{document}
\section*{Aufgabe 3}
Wir können die Menge B nicht trivial entscheiden da wir unendlich viele
Primzahlkombinationen durchgehen müssen um für eine Zahl sicher sagen zu
können das sie nicht in B liegt. \\
Wir definieren zunächst eine Hilfsfunktion $h'(x, t)$,
Das folgende Programm zeigt, wie h' berechnet wird.
\begin{lstlisting}[language=Python]
    """This program shows how to compute h'"""

def is_prime(x):
    return False if x < 2 else all([x % i != 0 for i in range(2, x - 1)])


def get_next_prime(x):
    x += 1
    while not is_prime(x):
        x += 1
    return x

def h(x, t):

    p = 2 
    q = 2

    for _ in range(t):
            
        if p - q == x:
            return x

        if p == q:
            q = 2
            p = get_next_prime(p)
        else:
            q += 1
    return 2

for curr in range(100):
    print(f" {curr}: {h(curr, 1000)}")
\end{lstlisting}
Nun können wir $h: \mathbb{N} \rightarrow \mathbb{N}$ definieren: \\
Wir benutzen die Cantorsche Paarungsfunktion so, dass wir ein gegbenes 
x in ihre beiden Komponenten zerlegen und als input für die Funktion h'(x, t)
benutzen. Da h'(x, t) immer stoppt ist die Funktion berechenbar (stoppt nach t
iterationen). Der Wertebereich ist passend da h'(x, t) für jedes x, falls es in b liegt,
für ein genügend großes t den richtigen Wert annimmt.
\end{document}