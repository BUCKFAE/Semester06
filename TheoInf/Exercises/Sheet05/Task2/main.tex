\documentclass[14pt]{article}

% Packages
\usepackage[utf8]{inputenc}
\usepackage[german]{babel}
\usepackage{amssymb}
\usepackage{fancyhdr}
\usepackage[framemethod=TikZ]{mdframed}
\usepackage{amsthm}
\usepackage{amsmath}
\usepackage[T1]{fontenc}
\usepackage{mathabx}

\title{Theoretische Informatik}
\author{Julian Schubert}

% Definition
\mdfdefinestyle{theoremstyle}{
    linecolor=blue!20,
    linewidth=2pt,
    frametitlerule=true,
    frametitlebackgroundcolor=gray!20,
    innertopmargin=\topskip
}
\mdtheorem[style=theoremstyle]{definition}{Definition}

% Eigenschaft
\mdfdefinestyle{eigenschaftstyle}{
    linecolor=red!50,
    linewidth=2pt,
    frametitlerule=true,
    frametitlebackgroundcolor=gray!20,
    innertopmargin=\topskip
}
\mdtheorem[style=eigenschaftstyle]{eigenschaft}{Eigenschaft}


% Kopf- / Fußzeile
\makeatletter
\let\runauthor\@author
\let\runtitle\@title
\pagestyle{fancy}
\fancyhf{}
\rhead{\runtitle}
\lhead{\runauthor}
\cfoot{\thepage}

\newcommand{\subsubsubsection}[1]{\paragraph{#1}\mbox{}\\}
\setcounter{secnumdepth}{4}
\setcounter{tocdepth}{4}


\begin{document}
\section*{Aufgabe 2}
\textbf{RE} bezeichnet alle Mengen die Aufzählbar sind, also Mengen A, für die
eine Funktion $h: \mathbb{N} \rightarrow \mathbb{N}^n$ mit $W_h$ = A existiert. \\
Da A und B in \textbf{RE} liegen, bedeutet dies, dass zwei Funktionen \\
$f(x) = \mathbb{N} \rightarrow \mathbb{N}^n$ mit $W_f = A$ und \\
$g(x) = \mathbb{N} \rightarrow \mathbb{N}^n$ mit $W_g = B$ existieren \\
Wir bilden damit nun wieder wie in Aufgabe 1 unsere Funktion h: \\
\begin{equation*}
    h(x) =
    \begin{cases}
        f(\lfloor x / 2\rfloor)                 & \text{falls $x = 0$ oder $x$ eine 
        gerade Zahl ist} \\
        g(\lfloor x / 2\rfloor)                 & \text{sonst}             
    \end{cases}
\end{equation*}
Da wir eine Funktion $h: \mathbb{N} \rightarrow \mathbb{N}^n$ mit $W_h = A \cup B$
angeben können, folgt: 
\[
    A, B \in RE \Rightarrow A \cup B \in RE 
\] 
\end{document}