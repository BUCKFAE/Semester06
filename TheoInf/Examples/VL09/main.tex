\documentclass{article}
% General document formatting
\usepackage[margin=0.7in]{geometry}
\usepackage[parfill]{parskip}
\usepackage[utf8]{inputenc}

\usepackage{listings}
\usepackage{amsmath}

% Related to math
\usepackage{amsmath,amssymb,amsfonts,amsthm}

\begin{document}
Anwendung Satz von Rice: \\
Sei $\varphi : \mathbb{N}^n \rightarrow \mathbb{N}$ berechenbar \\
$A = \{ i | M_i \text{ berechnet } \varphi\}$ \\
$A = \{ i | \text{ die von $M_i$ berechnete Funktion } \mathbb{N}^n
\rightarrow \mathbb{N} \text{ ist } \varphi\}$ \\
$A = \{ i | \text{ die von $M_i$ berechnete Funktion } \mathbb{N}^n
\rightarrow \mathbb{N} \text{ liegt in } S \}$ \\
$A = I(S)$, wobei $S = \{ \varphi \}$ \\
Es gilt: $S \neq \emptyset$ und $S \subsetneq \{ \psi | \psi 
\mathbb{N}^n \rightarrow \mathbb{N} \text{ berechenbar} \}$ \\
$Rightarrow$ Satz von Rice zeigt I(S) ist unentscheidbar, und somit auch A \\

Zweites Beispiel: VL 09 Minute 17
\end{document}

