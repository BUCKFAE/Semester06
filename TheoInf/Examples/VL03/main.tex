\documentclass{article}
% General document formatting
\usepackage[margin=0.7in]{geometry}
\usepackage[parfill]{parskip}
\usepackage[utf8]{inputenc}

\usepackage{listings}

% Related to math
\usepackage{amsmath,amssymb,amsfonts,amsthm}

\begin{document}
\textbf{Bestimmung der berechneten Funktion}
\begin{lstlisting}
    def f1(x, y):
        z = 0
        while (x > y):
            x = (x + 2)
            y = (y + 3)
            z = ( z + 1)
        return z
\end{lstlisting}

In jedem Schleifendurchlauf: 
\begin{itemize}
    \item x wird um 2 erhöht
    \item y wird um 3 erhöht
    \item z wird um 1 erhöht
\end{itemize}
$\Rightarrow$ Wert $x - y$ verringert sich in jedem Durchlauf
um 1 \\
\textbf{Schleifenabbruch:} falls $x - y \geq 0$ \\
$\Rightarrow$ Anzahl schleifendurchläufe beim Aufruf f1(a, b): \\
\begin{itemize}
    \item $a - b$ falls $a \geq b$
    \item $0$, sonst
\end{itemize}
$\Rightarrow$ Die Ausgabe $z$ ist die Anzahl der schleifendurchläufe \\
$\Rightarrow$ f(a, b) = a - b falls $a \geq b$, 0 sonst\\
\\
Anderes Beispiel: VL 03 ab Minute 33
\end{document}