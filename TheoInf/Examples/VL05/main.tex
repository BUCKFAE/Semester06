\documentclass{article}
% General document formatting
\usepackage[margin=0.7in]{geometry}
\usepackage[parfill]{parskip}
\usepackage[utf8]{inputenc}

\usepackage{listings}
\usepackage{amsmath}

% Related to math
\usepackage{amsmath,amssymb,amsfonts,amsthm}

\begin{document}
\section{Berechnung der k-adischen Darstellung}
\subsection{k = 2 (dyadische Darstellung)}

\begin{align*}
    109 &= 2 \cdot 54 + 1 \\
    54 &= 2 \cdot 26 + 2 \\
    26 &= 2 \cdot 12 + 2 \\
    12 &= 2 \cdot 5 + 2 \\
    5 &= 2 \cdot 2 + 1 \\
    2 &= 2 \cdot 0 + 2
\end{align*}
$\Rightarrow$ dya(109) = 212221 (von unten nach oben gelesen) \\
\textbf{Probe:} \\
$2 \cdot 2^5 +  1 \cdot 2^4 + 2 \cdot 2^3 + 2 \cdot 2^2 + 2 \cdot 2^1 + 1 \cdot 2^0 = 109$

\subsection{k = 3}
Beispiel: Vl 05 Minute 28
\end{document}