\documentclass{article}
% General document formatting
\usepackage[margin=0.7in]{geometry}
\usepackage[parfill]{parskip}
\usepackage[utf8]{inputenc}

\usepackage{listings}

% Related to math
\usepackage{amsmath,amssymb,amsfonts,amsthm}

\begin{document}
Aussage: \\
$
    \forall n \in \mathbb{N} - \{0, 1\} \exists p, q \in 
    \mathbb{N}-\{0, 1\}: \\
    2n = p +  q \land \forall x,y \in \mathbb{N} - \{0, 1\} (p \neq xy \land q \neq xy)
$
$\Rightarrow$ Jede Gerade Zahl größer als zwei kann als die Summe von zwei Primzahlen 
geschrieben werden \\
Es kann keinen Algorithmus geben der alle logische Aussagen (wie die hier gegebene) richtig
beantworten kann. \\
$\Rightarrow$ Goldbachsche Vermutung (noch ungeklärt) \\
Da wir die Lösung zur Goldbachschen Vermutung noch nicht kennen gibt es auch (noch) keinen 
Algorithmus der alle Logischen Aussagen lösen kann
\\
\\
Wir wissen auch nicht ob zwei Programme immer das selbe tun, 3n + 1 Vermutung
\end{document}