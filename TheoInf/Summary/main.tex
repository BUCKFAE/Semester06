\documentclass[14pt]{article}

% Packages
\usepackage[utf8]{inputenc}
\usepackage[german]{babel}
\usepackage{amssymb}
\usepackage{fancyhdr}
\usepackage[framemethod=TikZ]{mdframed}
\usepackage{amsthm}
\usepackage[T1]{fontenc}

\title{Theoretische Informatik}
\author{Julian Schubert}

% Theorem
\mdfdefinestyle{theoremstyle}{
    linecolor=blue!50,
    linewidth=2pt,
    frametitlerule=true,
    frametitlebackgroundcolor=gray!20,
    innertopmargin=\topskip
}
\mdtheorem[style=theoremstyle]{definition}{Definition}

% Kopf- / Fußzeile
\makeatletter
\let\runauthor\@author
\let\runtitle\@title
\pagestyle{fancy}
\fancyhf{}
\rhead{\runtitle}
\lhead{\runauthor}
\cfoot{\thepage}

\newcommand{\subsubsubsection}[1]{\paragraph{#1}\mbox{}\\}
\setcounter{secnumdepth}{4}
\setcounter{tocdepth}{4}


\begin{document}


    \maketitle
    \tableofcontents

    \newpage

    \section{Wichtige Vermutungen}
    \begin{definition}[Goldbachsche Vermutung]
        Jede natürliche gerade Zahl größer 2 ist Summe zweier Primzahlen
    \end{definition}
    \begin{definition}[Collaz-Problem (3n +1)-Vermutung]
        \begin{itemize}
            \item Beginne mit irgendeiner natürlichen Zahl $n > 0$
            \item Ist n gerade, son imm als nächstes $n // 2$ (abrundende Division)
            \item Wiederhole das Vorgehen mit der erhaltenen Zahl
        \end{itemize}
        \noindent
        \textbf{Vermutung:} Jede so konstruierte Zahlenfolge mündet in den Zyklus
            4, 2, 1, egal mit welcher natürlichen zahl $n > 0$ beginnt
    \end{definition}
    
    \section{Elementare Begriffe}
        \subsection{Komplexitätsklassen}
        \[
            ALL \subset P \subset NP
        \]
        \begin{itemize}
            \item \textbf{ALL:} Alle Probleme
            \item \textbf{NP:} Probleme, deren Lösungen schnell übrprüft weden können
                (effizient überprüfbare Probleme)
            \item \textbf{P:} Probleme, die isch in polynomieller Zeit lösen lassen 
                (effizient lösbare Probleme)
        \end{itemize}
        \subsection{Funktionen}
        % TODO: Folie 6, gehört bis 0:52
\end{document}