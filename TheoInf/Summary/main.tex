\documentclass[14pt]{article}

% Packages
\usepackage[utf8]{inputenc}
\usepackage[german]{babel}
\usepackage{amssymb}
\usepackage{fancyhdr}
\usepackage[framemethod=TikZ]{mdframed}
\usepackage{amsthm}
\usepackage{amsmath}
\usepackage[T1]{fontenc}
\usepackage{mathabx}

\title{Theoretische Informatik}
\author{Julian Schubert}

% Definition
\mdfdefinestyle{theoremstyle}{
    linecolor=blue!50,
    linewidth=2pt,
    frametitlerule=true,
    frametitlebackgroundcolor=gray!20,
    innertopmargin=\topskip
}
\mdtheorem[style=theoremstyle]{definition}{Definition}

% Eigenschaft
\mdfdefinestyle{eigenschaftstyle}{
    linecolor=red!50,
    linewidth=2pt,
    frametitlerule=true,
    frametitlebackgroundcolor=gray!20,
    innertopmargin=\topskip
}
\mdtheorem[style=eigenschaftstyle]{eigenschaft}{Eigenschaft}


% Kopf- / Fußzeile
\makeatletter
\let\runauthor\@author
\let\runtitle\@title
\pagestyle{fancy}
\fancyhf{}
\rhead{\runtitle}
\lhead{\runauthor}
\cfoot{\thepage}

\newcommand{\subsubsubsection}[1]{\paragraph{#1}\mbox{}\\}
\setcounter{secnumdepth}{4}
\setcounter{tocdepth}{4}


\begin{document}


    \maketitle
    \tableofcontents

    \newpage

    \section{Wichtige Vermutungen}
    \begin{definition}[Goldbachsche Vermutung]
        Jede natürliche gerade Zahl größer 2 ist Summe zweier Primzahlen
    \end{definition}
    \begin{definition}[Collaz-Problem (3n +1)-Vermutung]
        \begin{itemize}
            \item Beginne mit irgendeiner natürlichen Zahl $n > 0$
            \item Ist n gerade, so nimm als nächstes $n // 2$ (abrundende Division)
            \item ist n ungerade, so nimm als nächstes $3n + 1$
            \item Wiederhole das Vorgehen mit der erhaltenen Zahl
        \end{itemize}
        \noindent
        \textbf{Vermutung:} Jede so konstruierte Zahlenfolge mündet in den Zyklus
            4, 2, 1, egal mit welcher natürlichen zahl $n > 0$ beginnt
    \end{definition}
    \begin{definition}[Ackermann-Funktion]
        \textbf{Frage:} Gilt LOOP = $\{f \in$ WHILE | $f$ ist total\}? \\
        Die folgende Funktion (auch \textbf{Ackermann-Funktion} genannt)
        $a: \mathbb{N} \times \mathbb{N} \rightarrow \mathbb{N}$ ist
        total und While-berechenbar, aber nicht Loop-berechenbar: \\
        \begin{equation*}
            a(n, m) = 
            \begin{cases}
                m + 1 & \text{falls $n = 0$} \\
                a(n - 1, 1) & \text{falls $n > 0$ und $m = 0$} \\
                a(n - 1, a(n, m - 1)) & \text{falls $n > 0$ und $m > 0$}
            \end{cases}
        \end{equation*}
        $\Rightarrow$ Die Ackermann-Funktion ist eine totale Funktion in 
        $WHILE-LOOP$
    \end{definition}
    \section{Elementare Begriffe}

        \subsection{Komplexitätsklassen}
        \[
            ALL \subset P \subset NP
        \]
        \begin{itemize}
            \item \textbf{ALL:} Alle Probleme
            \item \textbf{NP:} Probleme, deren Lösungen schnell übrprüft weden können
                (effizient überprüfbare Probleme)
            \item \textbf{P:} Probleme, die isch in polynomieller Zeit lösen lassen 
                (effizient lösbare Probleme)
        \end{itemize}

        \subsection{Funktionen}
        \begin{definition}[Funktionen]
            Seien $f: A \rightarrow B$ und $g : B \rightarrow C$ Funktionen
            \begin{itemize}
                \item \textbf{Definitionsbereich} von f: \\
                $D_f = \{a \in A |$ es existiert ein $ b \in B$ mit $f(a) = b$\} \\
                $\Rightarrow$ Alles was etwas im Wertebereich trifft
                \item \textbf{Wertebereich} von f: \\
                $D_f = \{a \in A |$ es existiert ein $ a \in A$ mit $f(a) = b$\} \\
                $\Rightarrow$ alles was von etwas im Definitionsbereich 
                getroffen wird
                \item \textbf{Total:} $D_f = A$
                \item \textbf{Surjektiv:} $W_f = B$
                \item \textbf{Injektiv:} aus $a_1, a_2 \in D$ und $a_1 \neq a_2$
                folgt $f(a_1) \neq f(a_2)$
                \item \textbf{Bijektiv:} $f$ ist total, surjektiv und injektiv
                \item ist $f$ injektiv, so existiert die \textbf{Umkehrfunktion}
                $f^{-1}: B \rightarrow A$ mit $f^{-1}(b) = $ dasjenige $a  \in A$
                mit $f(a) = b$
            \end{itemize}
        \end{definition}

        \subsection{Binärdarstellung}
        \begin{definition}
            Jede natürliche Zahl $n \geq 1$ ist in genau einer Weise darstellbar
            als 
            \[
                n = \sum_{i = 0}^m a_i \cdot 2^i
            \]
            mit $m \in \mathbb{N}, a_m = 1$ und $a_0, \dots, a_{m - 1} 
            \in \{0, 1\}$.
        \end{definition}
        \begin{eigenschaft}[Binärdarstellung]
            \begin{center}
            $bin(2n + a) = bin(n)a$ für $n \geq 1$ und $a \in  \{0, 1\}$ 
            \end{center}
        \end{eigenschaft}

        \subsection{Listencodierung}
        Liste von Binärzahlen: $\langle x_1, \dots, x_n \rangle$ \\
        \textbf{Anwendung:} Bits verdoppeln, 10 alss Anfangs-, Trenn- und Enmarkierung
        \textbf{Beispiele:} \\
        $
            \indent \langle \rangle = bin^{-1}(10) = 2 \\
            \indent \langle 2 \rangle = bin^{-1}(10110010) = 178 \\
            \indent \langle 5, 3, 2 \rangle = bin^{-1}(1011001110111110110010) = 2944946
        $
    \section{While-Programme}
    \begin{definition}[While-Berechenbarkeit]
        Eine Funktion ist dann \textbf{While-Berechenbar}, falls es 
        ein While-Programm gibt, sodass der Definitionsbereich von 
        beiden identisch ist und der Wert für alle Eingaben übereinstimmt.
    \end{definition}
    \begin{definition}[Loop-Programm]
        ein \textbf{Loop-Programm} ist ein While-Programm mit folgenden
        Eigenschaften: 
        \begin{itemize}
            \item Das Programm enthält keine While-Schleifen
            \item Aus einer Funktion können nur weiter oben deklarierte
            Funktionen aufgerufen werden. Insbesondere sind keine 
            Selbstaufrufe erlaubt
            \item Das Programm enhält nur Funktionsdeklarationen mit 
            Initialiserung
            \item Das Programm ist für alle Eingaben definiert
        \end{itemize}
    \end{definition}
    $\Rightarrow$ Alle Loop-berechenbaren Funktionen sind total.
    \subsection{Berechnende Funktion bestimmen}
    \begin{enumerate}
        \item Schauen für welche Eingabe(n) die Schleife(n) wie oft 
        ausgeführt werden
        \item Schauen was sich mit jedem Schleifendurchlauf verändert
    \end{enumerate}


    \section{Ram-Programme}
    \begin{definition}[modifizierte Differenz]
        \begin{equation*}
            x \dotdiv y =  md(x, y)
            \begin{cases}
                x - y & \text{falls $x > y$} \\
                0 & \text{sonst}
            \end{cases}
        \end{equation*}
    \end{definition}
\end{document}