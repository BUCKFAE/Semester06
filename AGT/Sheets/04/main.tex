\documentclass[a4paper]{article}

\usepackage[utf8]{inputenc}
\usepackage[ngerman]{babel}
\usepackage{amsmath,amssymb}
\usepackage[german,vlined,longend]{algorithm2e}
\usepackage{graphicx}
\PassOptionsToPackage{usenames,dvipsnames,svgnames}{xcolor}  
\usepackage{tikz}
\usetikzlibrary{arrows,positioning,automata}
\usepackage{listings}

% --- math operators ---
\usepackage{mathtools}
\DeclarePairedDelimiter\set{\{}{\}} % use $\set*{1, 2, 3}$ 
\DeclarePairedDelimiter\abs{\lvert}{\rvert}
\DeclarePairedDelimiter\norm{\lVert}{\rVert}
\DeclarePairedDelimiter\ceils{\lceil}{\rceil}
\DeclarePairedDelimiter\floor{\lfloor}{\rfloor}
\DeclarePairedDelimiter\angles{\langle}{\rangle}
\def\Oh{\ensuremath{\mathcal{O}}} % big O like $\Oh(n)$
\def\oh{\ensuremath{\scriptstyle{\mathcal{O}}}} % small O
% ---

\begin{document}

\begin{small}
    \noindent
    Schubert Julian, Gruppe 4 \\
    Philipp Wahl, Gruppe 4
\end{small}
\bigskip

\begin{center}
    \LARGE Abgabe zum 4. Übungsblatt (AGT 21)
\end{center}
\smallskip
\subsection*{Aufgabe 1:}
$G_f$ = Residual Graph durch Ford  und  Fulkerson\\
${\displaystyle S\leftarrow \emptyset }$
  
T $\leftarrow \emptyset $

//möglich durch Breitensuche von s aus \\

Für jeden knoten $v \in V$ der von s in $G_f$ erreichbar ist:\\
\noindent\hspace*{10mm}füge v zu S hinzu \\
Für jeden weiteren Knoten $u \in V$:\\
\noindent\hspace*{10mm}
füge u zu T hinzu\\
C $\leftarrow \emptyset$\\
Für jede Kante uv $in$ E:\\
\noindent\hspace*{10mm} Wenn u $in$ S und v $in$ T ist, dann uv zu C hinzufügen\\
C ist die Menge der Kanten für einen minimalen Schnitt


Korrektheit: Ein minimaler Schnitt hat genau den Wert des Maximalen Flusses.
Das bedeutet, das wir einen maximalen Fluss f finden können, und dann jede Kante, bei welcher die Kapazität voll ausgelastet ist, als Kante für den Schnitt nehmen.\\
Eine Kante, bei der die Kapazität nicht voll ausgelastet ist, zählt mehr in f, als die Kapazität der Kante in den Schnitt zählt.\\
Da es aber ein maximaler Fluss ist, muss es eine für jede Kante, welche nicht voll ausgenutzt sein, irgendwann eine Kante folgen, welche komplett ausgenutzt ist, spätestens wenn wir zu t gehen.\\
Wenn eine Kante nun eine ungenutzte Kapazität hat, ist der Endknoten von s in $G_f$ erreichbar, also wissen wir, dass diese Kante keine Schnittkante sein kann.\\
Jede Kante, welche jedoch eine voll ausgenutzte Kapazität hat, zählt genau so in f, wie sie auch im Schnitt zählt. Der Endknoten einer solchen Kante ist nicht in $G_f$ erreichbar, und wird dementsprechend in unserem Algorithmus so hinzugefügt, dass die Kante ein Schnittkante wird.\\
\\
Laufzeit: wir benutzen sowohl den Algorithmus von Ford und Fulkerson, als auch die Breitensuche einmal. Dann gehen wir ein letztes mal durch alle Kanten durch. Das resultiert in:\\
O(V*$E^2$+V+E)
\subsection*{Aufgabe 2:}
\paragraph{a)}
"Da wir insgesamt 11 Knoten haben, und einer davon 10 Kanten hat, und wir einen einfachen Graphen haben, muss dieser Knoten eine Kante zu jedem anderem Knoten haben.
Weiterhin haben wir einen Knoten mit 9 Kanten, das heißt, dieser Knoten hat zu jedem anderem Knoten außer einem eine Kante.
Durch diese beiden Knoten haben alle außer einem Knoten im Graphen  mindestens 2 Kanten. Da wir jedoch vorgegeben haben, dass es 2 Knoten mit Grad 1 gibt, ist dieser Graph nicht möglich"
\paragraph{b)}
Wir modellieren das Problem wie folgt: Wir erstellen zunächst einen Knoten s und einen Knoten
t. Dann erstellen wir wie in der Skizze für n = 5 gezeigt jeden Knoten zwei mal (einmal mit
Index 1 und einmal mit index zwei). Dann Verbinen wir s mit jeden Knoten in der Ersten Spalte,
die Kantenkapazitäten sind dabei die entsprechenden Werte aus a. Dann wird jeder Knoten aus
der zweiten Spalte mit t verbunden, die Kantenkapazitäten sind hierbei die entsprechenden
Werte aus e. \\
Im letzen Schritt erstellen wir eine Kante von jedem Knoten aus der ersten Reihe zu jedem
Knoten aus der zweiten Reihe (außer wie an der Skizze gezeigt zu sich selbst). Die
Kapazität dieser Kanten ist dabe immer 1. In der Skizze haben wir dies zur Übersicht
nur für den Ersten Knoten durchgeführt. \\
Nun sucht man einen maximalen s-t-Fluss. Wenn
\[
    \sum_{a_i \in a} a_i = \textbf{ Wert des maximalen s-t-Flusses}
\]
gilt, so kann man einen Graphen mit der geforderten Eigenschaft konstruieren. \\
Die Korrektheit ergibt sich wie folgt: \\
Aus dem Knoten $v_i$ müssen genau $a_i$ Kanten rausführen. Wenn der Graph vollständig ist
ergibt sich $a_i = n - 1$, also werden alle Kantenkapazitäten voll ausgeschöpft. Wenn eine
Kante nicht benötigt wird (um die Flusserhaltung zu gewährleisten) wird die Kante im
s-t-Fluss nicht benutzt (0 / 1). Unser Konstrukt sorgt ebenfalls dafür, dass in dem Knoten
$v_i$ genau $e_i$ Kanten hinein führen (da $e_i$-viele Kanten die volle Kapazität nutzen). \\
Den entsprechenden Graphen kann man also Bilden, indem man für jede Kante von $v_{i1}$ nach
$v_{j2}$ schaut, ob diese die volle Kapazität (also 1) benutzt. Ist dies der Fall, wird
sie zum Graphen hinzugefügt. Da die Flusserhaltung nicht verletzt ist werden auch die
geforderten Eingangs- / Ausgangsgrade eingehalten.
\begin{figure}
    \begin{centering}

        \begin{tikzpicture}[>=stealth',shorten >=1pt,node distance=3cm,on grid,initial/.style    ={}]
            \node[state]          (s)                               {s};

            \node[state]          (v3)    [right =of s]     {$v3_1$};
            \node[state]          (v2)    [above =of v3]     {$v2_1$};
            \node[state]          (v1)    [above =of v2]     {$v1_1$};
            \node[state]          (v4)    [below =of v3]     {$v4_1$};
            \node[state]          (v5)    [below =of v4]     {$v5_1$};

            \node[state]          (v31)    [right =of v3]     {$v3_2$};
            \node[state]          (v21)    [right =of v2]     {$v2_2$};
            \node[state]          (v11)    [right =of v1]     {$v1_2$};
            \node[state]          (v41)    [right =of v4]     {$v4_2$};
            \node[state]          (v51)    [right =of v5]     {$v5_2$};

            \node[state]          (t)      [right =of v31]     {t};
            \tikzset{mystyle/.style={->,double=orange}}
            \tikzset{every node/.style={fill=white}}
            \tikzset{mystyle/.style={->,double=black}}

            \path (s)     edge [mystyle]   node   {$a_1$} (v1);
            \path (s)     edge [mystyle]   node   {$a_2$} (v2);
            \path (s)     edge [mystyle]   node   {$a_3$} (v3);
            \path (s)     edge [mystyle]   node   {$a_4$} (v4);
            \path (s)     edge [mystyle]   node   {$a_5$} (v5);

            \path (v1)     edge [mystyle]   node   {$1$} (v21);
            \path (v1)     edge [mystyle]   node   {$1$} (v31);
            \path (v1)     edge [mystyle]   node   {$1$} (v41);
            \path (v1)     edge [mystyle]   node   {$1$} (v51);

            \path (v11)     edge [mystyle]   node   {$e_1$} (t);
            \path (v21)     edge [mystyle]   node   {$e_2$} (t);
            \path (v31)     edge [mystyle]   node   {$e_3$} (t);
            \path (v41)     edge [mystyle]   node   {$e_4$} (t);
            \path (v51)     edge [mystyle]   node   {$e_5$} (t);

        \end{tikzpicture}
    \end{centering}
    \caption{Skizze für n = 5}
\end{figure}
\section*{Aufgabe 3}
\paragraph{a)}
\begin{figure}
    \begin{centering}
        \begin{tikzpicture}[>=stealth',shorten >=1pt,node distance=3cm,on grid,initial/.style    ={}]
            \node[state]          (s)                               {s};

            \node[state]          (v1)    [right =of s]     {$0$};
            \node[state]          (v2)    [below =of v1]     {$-10$};
            \node[state]          (v3)    [above =of v1]     {$10$};


            \node[state]          (t)      [right =of v1]     {t};
            \tikzset{mystyle/.style={->,double=orange}}
            \tikzset{every node/.style={fill=white}}
            \tikzset{mystyle/.style={->,double=black}}

            \path (s)     edge [mystyle]   node   {$10/10$} (v2);
            \path (v3)     edge [mystyle]   node   {$10/10$} (t);
            \path (v2)     edge [mystyle]   node   {$10/20$} (v1);
            \path (v1)     edge [mystyle]   node   {$10/20$} (v3);



        \end{tikzpicture}
        \caption{Skizze für zwei Knoten die sich ausgleichen}
    \end{centering}
\end{figure}
Ein Knoten mit negativem Bedarf bedeutet, dass er im Graphen G mehr abgeben kann als er aufnimmt.
Wir fügen zu unserem bestehenden Graphen G noch die Knoten s und t hinzu. Hat ein Knoten einen 
negativen Bearf, fügen wir eine Kante von s zu diesem Knoten hinzu. die Kapatzität der Kante 
ist dabei abs(bedarf). Hat ein Knoten einen positiven Bedarf fügen wir eine Kante von diesem 
Knoten zum Knoten t hinzu, die Kapazität ist hierbei wieder der Bedarf des Knotens. Gibt es 
nun einen s-t-Fluss mit Wert gleich der Summe aller Knotenkapazitäten so können alle Knoten 
mit negativem Bedarf die Knoten mit positivem Bedarf ausgleichen.
\paragraph{b)}
Unsere Lösung ist korrekt da wenn es einen wie oben konstruierten s-t-Fluss gibt, dann 
gibt es einen b-fluss. Die Kapatzitäten aller Kanten ist dabei genau die des orgeinalen Graphen
aus der Skizze.
\end{document}