\documentclass[a4paper]{article}

\usepackage[utf8]{inputenc}
\usepackage[ngerman]{babel}
\usepackage{amsmath,amssymb}
\usepackage[german,vlined,longend]{algorithm2e}
\usepackage{graphicx}
\PassOptionsToPackage{usenames,dvipsnames,svgnames}{xcolor}  
\usepackage{tikz}
\usetikzlibrary{arrows,positioning,automata}
\usetikzlibrary{calc}
\usepackage{listings}
\usepackage{float}

% --- math operators ---
\usepackage{mathtools}
\DeclarePairedDelimiter\set{\{}{\}} % use $\set*{1, 2, 3}$ 
\DeclarePairedDelimiter\abs{\lvert}{\rvert}
\DeclarePairedDelimiter\norm{\lVert}{\rVert}
\DeclarePairedDelimiter\ceils{\lceil}{\rceil}
\DeclarePairedDelimiter\floor{\lfloor}{\rfloor}
\DeclarePairedDelimiter\angles{\langle}{\rangle}
\def\Oh{\ensuremath{\mathcal{O}}} % big O like $\Oh(n)$
\def\oh{\ensuremath{\scriptstyle{\mathcal{O}}}} % small O
% ---

\begin{document}

\begin{small}
    \noindent
    Schubert Julian, Gruppe 4 \\
    Philipp Wahl, Gruppe 4
\end{small}
\bigskip

\begin{center}
    \LARGE Abgabe zum 10. Übungsblatt (AGT 21)
\end{center}
\smallskip
\subsection*{Aufgabe 1:}
\paragraph*{a)}
Angenommen eine Geradenüberdeckung enthält eine oder mehrere Geraden die jeweils
nur einen Punkt enthalten. Gibt es zwei (oder mehr) Geraden die nur einen Punkt enthalten
so können zwei Geraden die jeweils nur durch einen Punkt fürhren durch eine neue Gerade 
ersetzt werden, die durch die beiden Punkte führt. Gibt es nur eine Gerade die nur 
ein Punkt liegt dann kann man diese Gerade entfernen und durch eine neue Gerade ersetzen
die durch den alten Punkt und einen beliebigen neuen Punkt definiert ist. Daher genügt es
(für $n \geq 2$) Geraden zu betrachten, auf denen mindestens zwei Punkte liegen
\paragraph*{b)}
Dann muss diese Gerade in der Geradenüberdeckung enthalten sein, da wenn man zwei Punkte
auf dieser Gerade verbindet automatisch alle anderen Punkte auf der Geraden liegen. Wäre
diese Gerade dementsprechend nicht in der Überdeckung so müsste es für jeden der k Punkte
auf der ursprünglichen Gerade eine neue Gerade existieren, es müssten also mehr als k 
Geraden existieren würde man die Gerade auf der mehr als k Punkte liegen nicht mit aufnehmen.
Daher müssen alle Geraden die mehr als k Punkte haben ins Matching aufgenommen werden.
\paragraph*{c)}
Wenn es keine Gerade gibt die mehr als k Punkte enthält folgt daraus, dass jede der 
k vielen Geraden maximal k Punkte enthält. Insgesamt können wir so also mit k Geraden
auf denen jeweils maximal k viele Punkte liegen $k^2$ viele Punkte Abdecken. \\
Gilt also $k^2 < n$ und es gibt keine Gerade die mehr als k Punkte enthält, so kann
es keine Geradenüberdeckung geben.
\subsection*{Aufgabe 2:}
\paragraph*{a)}
In einem Graphen $G = (V, E)$ mit $n := |V|$ gibt es $2^n$ mögliche Knoten-Teilmengen.
Wir generieren alle möglichen Teilmengen, Prüfen für jede Teilmenge ob sie eine Clique ist
(indem wir für jeden Knoten in der Clique alle Nachbarn prüfen),
und speichern uns immer die größte bisher gefundene Clique. Damit ergibt sich eine Laufzeit
von $\Oh(2^n \cdot V^2)$
\paragraph*{b)}
Es kann keine Clique mit mehr als $\Delta$ vielen Knoten geben da in einer Clique alle
Knoten zueinande Benachbart sind. Hätte eine Clique also mehr als $\Delta$ viele Knoten 
so wäre der $\Delta$ ebenfalls größer \\
Wir können nun also alle Knoten-Teilmengen bilden die höchstens $\Delta$ viele Knoten
enthalten. Dann können wir ebenfalls wieder für jede dieser Teilmenge bestimmen ob 
sie eine Clique ist und uns die größte Merken. Wir erhalten eine Laufzeit von 
$\Oh(2^{\Delta} \cdot V^2)$ 
\paragraph*{c)}
Wir können nun für jeden Knoten annehmen das er Teil der Größten Clique ist. Dann bestimmen
wir für jeden Nachbarn von diesem Knoten plus dem Knoten selbst den größten Grad und wenden
dann den Algorithmus aus Teilaufgabe b an. Dadurch ergibt sich eine Laufzeit von 
$\Oh(2^{\Delta} \cdot V^2)$
\subsection*{Aufgabe 3:}
\paragraph*{a)}
Wir zeigen das es keinen größeren Fluss geben kann als der von uns in Abbildung \ref{3a}
gezeichnete.
\begin{figure}[H]
    \centering
    \begin{tikzpicture}[>=stealth',shorten >=1pt,node distance=3cm,on grid,initial/.style,     ={}]
            \node[state]          (v1)                          {1};
            \node[state]          (v2) [below left  =of v1]     {2};
            \node[state]          (v3) [below right =of v1]     {3};

            \node[state]          (s1) [above left  =of v1]     {$s_1$};
            \node[state]          (t2) [above right =of v1]     {$t_2$};

            \node[state]          (t3) [left        =of v2]     {$t_3$};
            \node[state]          (s3) [right       =of v3]     {$s_3$};

            \node[state]          (s2) [below       =of v2]     {$s_2$};
            \node[state]          (t1) [below       =of v3]     {$t_1$};

            \tikzset{mystyle/.style={->,double=orange}}
            \tikzset{every node/.style={fill=white}}
            \tikzset{mystyle/.style={->,double=black}}
 
            \path (s1)     edge [mystyle]  node  {1 / 1} (v1);
            \path (v1)     edge [mystyle]  node  {1 / 1} (v2);
            \path (v2)     edge [mystyle]  node  {1 / 1} (v3);
            \path (v3)     edge [mystyle]  node  {1 / 1} (t1);

            \path (v1)     edge [mystyle]  node  {0 / 1} (t2);
            \path (v3)     edge [mystyle]  node  {0 / 1} (v1);
            \path (v2)     edge [mystyle]  node  {0 / 1} (t3);
            \path (s3)     edge [mystyle]  node  {0 / 1} (v3);
            \path (s2)     edge [mystyle]  node  {0 / 1} (v2);



    \end{tikzpicture}
    \caption{Maximaler Gesamtflusswert: 1}
    \label{3a}
\end{figure}
Setzt man eine der andern von einem s ausgehenden Knoten auf 1 so erhält man ebenfalls
einen Maximalen Gesamtflusswert von 1 (da sich die Flüsse gegenseitig Blockieren)
\paragraph*{b)}
\begin{figure}[H]
    \centering
    \begin{tikzpicture}[>=stealth',shorten >=1pt,node distance=3cm,on grid,initial/.style,     ={}]
            \node[state]          (v1)                          {1};
            \node[state]          (v2) [below left  =of v1]     {2};
            \node[state]          (v3) [below right =of v1]     {3};

            \node[state]          (s1) [above left  =of v1]     {$s_1$};
            \node[state]          (t2) [above right =of v1]     {$t_2$};

            \node[state]          (t3) [left        =of v2]     {$t_3$};
            \node[state]          (s3) [right       =of v3]     {$s_3$};

            \node[state]          (s2) [below       =of v2]     {$s_2$};
            \node[state]          (t1) [below       =of v3]     {$t_1$};

            \tikzset{mystyle/.style={->,double=orange}}
            \tikzset{every node/.style={fill=white}}
            \tikzset{mystyle/.style={->,double=black}}
 

            \path (v1)     edge [mystyle]  node  {1 / 1} (v2);
            \path (v2)     edge [mystyle]  node  {1 / 1} (v3);
            \path (v3)     edge [mystyle]  node  {1 / 1} (v1);

            \path (v3)     edge [mystyle]  node  {0.5 / 1} (t1);
            \path (s1)     edge [mystyle]  node  {0.5 / 1} (v1);
            \path (v1)     edge [mystyle]  node  {0.5 / 1} (t2);
            \path (v2)     edge [mystyle]  node  {0.5 / 1} (t3);
            \path (s3)     edge [mystyle]  node  {0.5 / 1} (v3);
            \path (s2)     edge [mystyle]  node  {0.5 / 1} (v2);

    \end{tikzpicture}
    \caption{Maximaler Gesamtflusswert: 1.5}
    \label{3b}
\end{figure}
Auch hier haben wir offensichtlich den maximalen Gesamtflusswert gefunden da die Kanten
$\{1, 2\}, \{2, 3\}$ und $\{3, 1\}$ voll ausgelastet sind.
\end{document}