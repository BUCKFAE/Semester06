\documentclass[a4paper]{article}

\usepackage[utf8]{inputenc}
\usepackage[ngerman]{babel}
\usepackage{amsmath,amssymb}
\usepackage[german,vlined,longend]{algorithm2e}
\usepackage{graphicx}
\PassOptionsToPackage{usenames,dvipsnames,svgnames}{xcolor}  
\usepackage{tikz}
\usetikzlibrary{arrows,positioning,automata}
\usetikzlibrary{calc}
\usepackage{listings}
\usepackage{float}

% --- math operators ---
\usepackage{mathtools}
\DeclarePairedDelimiter\set{\{}{\}} % use $\set*{1, 2, 3}$ 
\DeclarePairedDelimiter\abs{\lvert}{\rvert}
\DeclarePairedDelimiter\norm{\lVert}{\rVert}
\DeclarePairedDelimiter\ceils{\lceil}{\rceil}
\DeclarePairedDelimiter\floor{\lfloor}{\rfloor}
\DeclarePairedDelimiter\angles{\langle}{\rangle}
\def\Oh{\ensuremath{\mathcal{O}}} % big O like $\Oh(n)$
\def\oh{\ensuremath{\scriptstyle{\mathcal{O}}}} % small O
% ---

\begin{document}

\begin{small}
    \noindent
    Schubert Julian, Gruppe 4 \\
    Philipp Wahl, Gruppe 4
\end{small}
\bigskip

\begin{center}
    \LARGE Abgabe zum 10. Übungsblatt (AGT 21)
\end{center}
\smallskip
\subsection*{Aufgabe 3:}
\paragraph*{a)}
Wir zeigen das es keinen größeren Fluss geben kann als der von uns in Abbildung \ref{3a}
gezeichnete.
\begin{figure}[H]
    \centering
    \begin{tikzpicture}[>=stealth',shorten >=1pt,node distance=3cm,on grid,initial/.style,     ={}]
            \node[state]          (v1)                          {1};
            \node[state]          (v2) [below left  =of v1]     {2};
            \node[state]          (v3) [below right =of v1]     {3};

            \node[state]          (s1) [above left  =of v1]     {$s_1$};
            \node[state]          (t2) [above right =of v1]     {$t_2$};

            \node[state]          (t3) [left        =of v2]     {$t_3$};
            \node[state]          (s3) [right       =of v3]     {$s_3$};

            \node[state]          (s2) [below       =of v2]     {$s_2$};
            \node[state]          (t1) [below       =of v3]     {$t_1$};

            \tikzset{mystyle/.style={->,double=orange}}
            \tikzset{every node/.style={fill=white}}
            \tikzset{mystyle/.style={->,double=black}}
 
            \path (s1)     edge [mystyle]  node  {1 / 1} (v1);
            \path (v1)     edge [mystyle]  node  {1 / 1} (v2);
            \path (v2)     edge [mystyle]  node  {1 / 1} (v3);
            \path (v3)     edge [mystyle]  node  {1 / 1} (t1);

            \path (v1)     edge [mystyle]  node  {0 / 1} (t2);
            \path (v3)     edge [mystyle]  node  {0 / 1} (v1);
            \path (v2)     edge [mystyle]  node  {0 / 1} (t3);
            \path (s3)     edge [mystyle]  node  {0 / 1} (v3);
            \path (s2)     edge [mystyle]  node  {0 / 1} (v2);



    \end{tikzpicture}
    \caption{Maximaler Gesamtflusswert: 1}
    \label{3a}
\end{figure}
\paragraph*{b)}
\begin{figure}[H]
    \centering
    \begin{tikzpicture}[>=stealth',shorten >=1pt,node distance=3cm,on grid,initial/.style,     ={}]
            \node[state]          (v1)                          {1};
            \node[state]          (v2) [below left  =of v1]     {2};
            \node[state]          (v3) [below right =of v1]     {3};

            \node[state]          (s1) [above left  =of v1]     {$s_1$};
            \node[state]          (t2) [above right =of v1]     {$t_2$};

            \node[state]          (t3) [left        =of v2]     {$t_3$};
            \node[state]          (s3) [right       =of v3]     {$s_3$};

            \node[state]          (s2) [below       =of v2]     {$s_2$};
            \node[state]          (t1) [below       =of v3]     {$t_1$};

            \tikzset{mystyle/.style={->,double=orange}}
            \tikzset{every node/.style={fill=white}}
            \tikzset{mystyle/.style={->,double=black}}
 

            \path (v1)     edge [mystyle]  node  {1 / 1} (v2);
            \path (v2)     edge [mystyle]  node  {1 / 1} (v3);
            \path (v3)     edge [mystyle]  node  {1 / 1} (v1);

            \path (v3)     edge [mystyle]  node  {0.5 / 1} (t1);
            \path (s1)     edge [mystyle]  node  {0.5 / 1} (v1);
            \path (v1)     edge [mystyle]  node  {0.5 / 1} (t2);
            \path (v2)     edge [mystyle]  node  {0.5 / 1} (t3);
            \path (s3)     edge [mystyle]  node  {0.5 / 1} (v3);
            \path (s2)     edge [mystyle]  node  {0.5 / 1} (v2);

    \end{tikzpicture}
    \caption{Maximaler Gesamtflusswert: 1.5}

    \label{3b}
\end{figure}

\end{document}