\documentclass[a4paper]{article}

\usepackage[utf8]{inputenc}
\usepackage[ngerman]{babel}
\usepackage{amsmath,amssymb}
\usepackage[german,vlined,longend]{algorithm2e}
\usepackage{graphicx}
\PassOptionsToPackage{usenames,dvipsnames,svgnames}{xcolor}  
\usepackage{tikz}
\usetikzlibrary{arrows,positioning,automata}
\usepackage{listings}
\usepackage{float}

% --- math operators ---
\usepackage{mathtools}
\DeclarePairedDelimiter\set{\{}{\}} % use $\set*{1, 2, 3}$ 
\DeclarePairedDelimiter\abs{\lvert}{\rvert}
\DeclarePairedDelimiter\norm{\lVert}{\rVert}
\DeclarePairedDelimiter\ceils{\lceil}{\rceil}
\DeclarePairedDelimiter\floor{\lfloor}{\rfloor}
\DeclarePairedDelimiter\angles{\langle}{\rangle}
\def\Oh{\ensuremath{\mathcal{O}}} % big O like $\Oh(n)$
\def\oh{\ensuremath{\scriptstyle{\mathcal{O}}}} % small O
% ---

\begin{document}

\begin{small}
    \noindent
    Schubert Julian, Gruppe 4 \\
    Philipp Wahl, Gruppe 4
\end{small}
\bigskip

\begin{center}
    \LARGE Abgabe zum 7. Übungsblatt (AGT 21)
\end{center}
\smallskip
\subsection*{Aufgabe 1:}
\paragraph*{a)}
In einem s-Wurzelspannbaum ist jeder Knoten im Graphen
vom Knoten s aus erreichbar. Damit gilt $E(s) = V$
\paragraph*{b)}
Die Tiefensuche durchläuft einen Pfad immer so weit 
wie möglich, und springt dann erst zu einem Knoten zurück
der noch Nachbarn hat, die noch nicht besucht wurden.
So entsteht ein s-Wurzelspannbaum (falls $E(s) = V$ gilt) 
und man die Tiefensuche am Knoten s startet.
\paragraph*{c)}
Angenommen G ist Kreisfrei und es existieren $s_1, s_2$ mit 
$s_1 \neq s_2$ und es gibt zwei verschiedene Wurzelspannbäume,
einen mit $s_1$ als Wurzel und einen mit $s_2$ als Wurzel.
Dies bedeutet dann das es in G einen gerichteten Weg
von $s_1$ nach $s_2$ \textbf{und} einen gerichteten Weg
von $s_2$ nach $s_1$ gibt. Damit ist G nicht kreisfrei. \\
$\Rightarrow$ Die Aussage ist korrekt.
\paragraph*{d)}
Für den Graphen in G lassen sich die zwei skizzierten (unterschiedlichen)
Wurzellspannbäume bilden, die Aussage ist also falsch.
\begin{figure}[H]
    \centering
    \begin{tikzpicture}[>=stealth',shorten >=1pt,node distance=3cm,on grid,initial/.style    ={}]
            \node[state]          (s)                        {s};
            \node[state]          (a) [below left =of s]    {a};
            \node[state]          (b) [below right  =of s]    {b};
            \node[state]          (c) [below left =of b]    {c};

            \tikzset{mystyle/.style={->,double=black}}   
            \path (s)     edge [mystyle]  node  {} (a);
            \path (s)     edge [mystyle]  node  {} (b);
            \path (a)     edge [mystyle]  node  {} (c);
            \path (b)     edge [mystyle]  node  {} (c);
        \end{tikzpicture}
\caption{Graph G=(V, E)}
\end{figure}
\begin{figure}[H]
    \centering
    \begin{tikzpicture}[>=stealth',shorten >=1pt,node distance=3cm,on grid,initial/.style    ={}]
            \node[state]          (s)                        {s};
            \node[state]          (a) [below left =of s]    {a};
            \node[state]          (b) [below right  =of s]    {b};
            \node[state]          (c) [below left =of b]    {c};

            \tikzset{mystyle/.style={->,double=orange}}   
            \path (s)     edge [mystyle]  node  {} (a);
            \path (s)     edge [mystyle]  node  {} (b);
            \path (a)     edge [mystyle]  node  {} (c);
        \end{tikzpicture}
\caption{Möglicher Wurzelspannbaum 1}
\end{figure}
\begin{figure}[H]
\centering
        \begin{tikzpicture}[>=stealth',shorten >=1pt,node distance=3cm,on grid,initial/.style    ={}]
            \node[state]          (s)                        {s};
            \node[state]          (a) [below left =of s]    {a};
            \node[state]          (b) [below right  =of s]    {b};
            \node[state]          (c) [below left =of b]    {c};

            \tikzset{mystyle/.style={->,double=orange}}   
            \path (s)     edge [mystyle]  node  {} (a);
            \path (s)     edge [mystyle]  node  {} (b);
            \path (b)     edge [mystyle]  node  {} (c);
        \end{tikzpicture}
\caption{Möglicher Wurzelspannbaum 2}
\end{figure}
\subsection*{Aufgabe 2:}
Hinrichtung $\Rightarrow$: \\
Wenn der zugrunde liegende ungerichtete Graph ein Baum ist und indeg($v$) = 1 
für $v \neq s$ und indeg($s$) = 0 gilt, dann muss damit der Graph ein s-Wurzelspannbaum
ist nur noch gezeigt werden, dass der Graph Azyklisch ist, und er alle Knoten $v \in V  \backslash \{s\}$
enthält: \\
Jeder Knoten ist in dem neuen Graphen enthalten, dies ist gegeben da der indeg($v$) = 1
für jeden Knoten $v \in V  \backslash \{s\}$ ist, jeder Knoten wird also genau ein mal besucht. \\
Ebenso ist der neue Grpah azyklisch aufgrund dieser Eigenschaft. Würde der neue
Graph einen Kreis enthalten müsste entweder indeg($s$) $\geq 1$ oder indeg($v$) $geq 2$
für mindestens einen Knoten $v \in V \backslash \{s\}$ gelten (anders kann kein Kreis entstehen). \\
\\
Rückrichtung $\Leftarrow$: \\
Wenn G ein Wurzelspannbaum ist, dann ist er ein Wurzelbaum der alle Knoten enthält. \\
Demnach Gilt also das G (und damit der zugrunde liegende Graph) azyklisch ist und
für jeden Knoten $v \in V \backslash \{s\}$ gilt indeg($v$) = 1 (gegeben durch Definition
Wurzelbaum), und indeg($s$) = 0.
\subsection*{Aufgabe 3:}
\paragraph*{a)}
\begin{figure}[H]
    \centering
    \begin{tikzpicture}[>=stealth',shorten >=1pt,node distance=3cm,on grid,initial/.style    ={}]
            \node[state]          (s)                        {s};
            \node[state]          (a) [below left =of s]    {a};
            \node[state]          (b) [below right  =of s]    {b};
            \node[state]          (c) [below left =of b]    {c};

            \tikzset{mystyle/.style={->,double=black}}   
            \path (s)     edge [mystyle]  node  {} (a);
            \path (s)     edge [mystyle]  node  {} (b);
            \path (a)     edge [mystyle]  node  {} (c);
            \path (b)     edge [mystyle]  node  {} (c);
        \end{tikzpicture}
\caption{Hier schlägt Kruskal fehl}
\end{figure}
Die Zeichung zeigt einen Graphen in dem Kruskal fehlschlägt. Hier könnte Kruskal beim Knoten c starten,
da c aber keine ausgehenden Kanten hat kann c nicht root-element eines s-Wurzelspannbaums sein.
\paragraph*{b)}
\begin{figure}[H]
    \centering
    \begin{tikzpicture}[>=stealth',shorten >=1pt,node distance=3cm,on grid,initial/.style    ={}]
            \node[state]          (s)                           {s};
            \node[state]          (1) [below left =of s]        {1};
            \node[state]          (2) [below right  =of s]      {2};
            \node[state]          (3) [below =of 1]        {3};
            \node[state]          (4) [below  =of 2]        {4};

            \tikzset{mystyle/.style={->,double=orange}}
            \tikzset{every node/.style={fill=white}}
            \tikzset{mystyle/.style={->,double=black}}   
            \path (s)     edge [mystyle]  node  {4} (1);
            \path (s)     edge [mystyle]  node  {1} (2);
            \path (1)     edge [mystyle]  node  {2} (3);
            \path (1)     edge [mystyle]  node  {1} (4);
            \path (4)     edge [mystyle]  node  {3} (3);
            \path (2)     edge [mystyle]  node  {2} (4);

    \end{tikzpicture}
    \caption{Graph G = (V, E)}
\end{figure}
Hier findet Prim folgenden s-Wurzelspannbaum (ausgehend von s)
\begin{figure}[H]
    \centering
    \begin{tikzpicture}[>=stealth',shorten >=1pt,node distance=3cm,on grid,initial/.style    ={}]
            \node[state]          (s)                           {s};
            \node[state]          (1) [below left =of s]        {1};
            \node[state]          (2) [below right  =of s]      {2};
            \node[state]          (3) [below =of 1]        {3};
            \node[state]          (4) [below  =of 2]        {4};

            \tikzset{mystyle/.style={->,double=orange}}
            \tikzset{every node/.style={fill=white}}
            \tikzset{mystyle/.style={->,double=orange}}   
            \path (s)     edge [mystyle]  node  {4} (1);
            \path (s)     edge [mystyle]  node  {1} (2);
            \path (4)     edge [mystyle]  node  {3} (3);
            \path (2)     edge [mystyle]  node  {2} (4);

    \end{tikzpicture}
    \caption{s-Wurzelspannbaum mit Prim}
\end{figure}
Jedoch gibt es einen günstigeren s-Wuzrelspannbaum:
\begin{figure}[H]
    \centering
    \begin{tikzpicture}[>=stealth',shorten >=1pt,node distance=3cm,on grid,initial/.style    ={}]
            \node[state]          (s)                           {s};
            \node[state]          (1) [below left =of s]        {1};
            \node[state]          (2) [below right  =of s]      {2};
            \node[state]          (3) [below =of 1]        {3};
            \node[state]          (4) [below  =of 2]        {4};

            \tikzset{mystyle/.style={->,double=orange}}
            \tikzset{every node/.style={fill=white}}
            \tikzset{mystyle/.style={->,double=orange}}   
            \path (s)     edge [mystyle]  node  {4} (1);
            \path (s)     edge [mystyle]  node  {1} (2);
            \path (1)     edge [mystyle]  node  {2} (3);
            \path (1)     edge [mystyle]  node  {1} (4);

    \end{tikzpicture}
    \caption{kleinerer s-Wurzelspannbaum}
\end{figure}
Prim findet also selbst unter gegebenen Anweisungen nicht immer einen minimalen s-Wurzelspannbaum
\paragraph*{c)}
Der Gegebene Graph ist kreisfrei, wir müssen also nur noch zeigen das indeg($s$) = 0 und 
indeg($v$) = 1 für jeden Knoten $v \in V \backslash \{s\}$ gilt: \\
Der Algorithmus fügt in jeder Iteration eine Kante hinzu, die zum aktuellen Knoten $v$
hinführt. Da dies für jeden Knoten $v \in V \backslash \{s\}$ genau eingmal gemacht wird ist 
indeg($v$) = 1 für jeden Knoten $v \in V \backslash \{s\}$. Da keine Kante vom zum Knoten 
$s$ hin hinzugefügt wird (nur eventuell Kanten die von $s$ weg gehen) ist indeg($s$) = 0
ebenfalls erfüllt.
\paragraph*{d)}  
Der Wurzelspannbaum ist minimal. Dies muss gelten da wir für jeden Knoten die billigste Kante 
wählen die den Knoten zum Baum hinzufügt. Da wir wenn wir die billigste Kante für einen Knoten
wählen wir uns keine andere, bessere, Lösung verbauen können muss die Lösung optimal sein (da 
wir für jeden Knoten die beste Möglichkeit wählen und die Wahl für einen Knoten die Wahl für
keinen anderen Knoten einschränkt).
\end{document}