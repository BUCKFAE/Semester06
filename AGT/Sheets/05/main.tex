\documentclass[a4paper]{article}

\usepackage[utf8]{inputenc}
\usepackage[ngerman]{babel}
\usepackage{amsmath,amssymb}
\usepackage[german,vlined,longend]{algorithm2e}
\usepackage{graphicx}
\PassOptionsToPackage{usenames,dvipsnames,svgnames}{xcolor}  
\usepackage{tikz}
\usetikzlibrary{arrows,positioning,automata}
\usepackage{listings}

% --- math operators ---
\usepackage{mathtools}
\DeclarePairedDelimiter\set{\{}{\}} % use $\set*{1, 2, 3}$ 
\DeclarePairedDelimiter\abs{\lvert}{\rvert}
\DeclarePairedDelimiter\norm{\lVert}{\rVert}
\DeclarePairedDelimiter\ceils{\lceil}{\rceil}
\DeclarePairedDelimiter\floor{\lfloor}{\rfloor}
\DeclarePairedDelimiter\angles{\langle}{\rangle}
\def\Oh{\ensuremath{\mathcal{O}}} % big O like $\Oh(n)$
\def\oh{\ensuremath{\scriptstyle{\mathcal{O}}}} % small O
% ---

\begin{document}

\begin{small}
    \noindent
    Schubert Julian, Gruppe 4 \\
    Philipp Wahl, Gruppe 4
\end{small}
\bigskip

\begin{center}
    \LARGE Abgabe zum 5. Übungsblatt (AGT 21)
\end{center}
\smallskip
\subsection*{Aufgabe 1:}
\paragraph{a)}
Wir betrachten zunächst alle Blätter im Baum. gibt es ein perfektes Matching, so 
sind alle Knoten (also auch Blätter) teil dieses Matchings (bzw es gibt Kanten
zu diesen Knoten). Es muss also, wenn es ein perfektes Matching gibt, jede Kante
die einen Elternknoten mit einem Blatt verbindet im Matching enthalten sein (sonst
wäre das Blatt ja nicht Teil des Matchings). Das bedeutet aber auch, dass eine Kante,
die einen Knoten $v$ mit einem Knoten $u$ verbindet, wobei der Knoten $u$ Elternknoten
eines Blattes $b$ ist, nicht im Matching enthalten sein darf (da sonst zwei Kanten
den Start/Endpunkt $u$ hätten). Da $v$ aber auch im Matching enthalten sein muss, 
kann man nun diesen Knoten als Blatt betrachten (man lässt den Teilbaum unter 
selbigem Knoten einfach wegfallen). Gibt es ein perfektes Matching, kann als 
einzige Kante wieder nur die Kante die $u$ mit seinem Elternknoten verbindet gewählt
werden. Gibt es im Baum also ein perfektes Matching, muss dieses Eindeutig sein da 
es immer nur eine Möglichkeit gibt einen Knoten zum Matching hinzuzufügen.
\paragraph{b)}
Wir benutzen folgenden Algorighmus \\ \\
\noindent BerechneMatching($G$)\\
\begin{algorithm}[H]
    blätter = new Queue
    matching = $\emptyset$ \\
    \ForEach(Alle Blätter zur Liste hinzufügen){$v \in V$}{
        \If(Wenn der aktuelle Knoten $v$ ein Blatt ist, wird er
        zur Liste hinzugefügt){$v$ is Leaf}{
            blätter.Enqueue($v$)
        }
    }
    \While(Matching für alle Blätter bilden){not blätter.Empty()}{
        curr = blätter.Dequeue() \\
        parent = curr.parent \\
        // Kante vom Parent zum aktuellen Knoten zum matching hinzufügen \\
        matching = matching $\cup$ \{curr, parent\} \\
        // Parent vom Parent wird nun zur Blätter-Liste hinzugefügt \\
        blätter.Enqueue(parent.parent) \\
    }
    \Return matching \\
\end{algorithm}
\noindent\textbf{Laufzeit:} Das finden aller initialen Blätter im Graphen benötigt \Oh(V) (da wir
über alle Knoten iterieren und für jeden Knoten prüfen ob er ein Blatt ist). \\
Dann wird jeder Knoten maximal ein mal in der zweiten While-Schleife behandelt (es wird
jeder Zweite Knoten als curr ausgewählt). Die Operationen in der zweiten Schleife haben jeweils
Eine Laufzeit von \Oh(1). Die zweite Schleife hat somit also einen Laufzeit von \Oh(V). 
Insgesamt ergibt sich so für unseren Algorighmus eine Laufzeit von \Oh(V) (Da \Oh(V + V) = \Oh(V)) \\
\textbf{Korrektheit:}
Wir haben in Aufgabe a gezeigt, dass es für ein perfektes Matching immer nur eine Möglichkeit
gibt einen Knoten hinzuzufügen, da unser algorithmus nach genau dieser Vorgehensweise arbeitet,
ist unser algorithmus für perfekte Matchings korrekt. \\
Im falle eines nicht-perfekten-matchings findet unser Algorithmus ein größtes Matching. Dies
ist gegeben da wir so viele Blätter wie möglich auswählen (und somit auch so viele Knoten
wie möglich). 
\paragraph{c)}
Wir betrachten jeden Knoten sowie alle Kanten die zu diesem Knoten führen (bzw auch diese die
von dem Knoten weg führen). Wir können uns nun für jeden dieser Knoten die Kanten nach 
Gewicht sortieren (Das geht in \Oh(E log E) da wir jede Kante maximal zwei mal behandeln und das 
einfügen in die Liste an die richtige Stelle in \Oh(log E) funktioniert). Wir wählen dann die 
Kanten wie folgt aus: Ist eine Kante für zwei Knoten direkt die beste Kante für beide Knoten so 
wählen wir diese Kante. Sonst nehmen wir einfach die beste Übereinstimmende Kante. Das sorgt 
dafür, das jeder Knoten die beste Kante für sich wählt, außer wenn wir dadurch eine bessere 
Kombination kaputt machen würden, dann wählen wir die bessere Kombination. Dieses Vorgehen 
dauert \Oh($E^2$), insgesamt haben wir also eine Laufzeit von \Oh($E^2$)
\subsection*{Aufgabe 2:}
\paragraph{a)}
Wenn es drei Knoten gibt muss gelten: Der Knoten $v_2$ muss vom Knoten $v_1$ aus erreichbar sein 
(dies ist gegeben da der Graph stark zusammenhängend ist). Ebenso muss $v_3$ vom Knoten $v_2$
erreichbar sein und $v_1$ vom Knoten $v_4$. Es gibt also einen Kreis der Länge größer gleich 
3, der von $v_1$ (über k andere Knoten) nach $v_2$, von dort aus nach $v_3$ und zum Schluss wieder
nach $v_1$ führt. 
\paragraph{b)}
Wir gehen alle Knoten im Kreis durch der Entfernt werden kann ohne das die 
Bedingung verletzt wird, das können wir machen bis nur noch 3 Übrig.
Der Knoten muss existieren weil wegen Tunier und andere Bedingung
\paragraph{c)}
\end{document}