\documentclass[a4paper]{article}

\usepackage[utf8]{inputenc}
\usepackage[ngerman]{babel}
\usepackage{amsmath,amssymb}
\usepackage[german,vlined,longend]{algorithm2e}
\usepackage{graphicx}
\PassOptionsToPackage{usenames,dvipsnames,svgnames}{xcolor}  
\usepackage{tikz}
\usetikzlibrary{arrows,positioning,automata}
\usetikzlibrary{calc}
\usepackage{listings}
\usepackage{float}

% --- math operators ---
\usepackage{mathtools}
\DeclarePairedDelimiter\set{\{}{\}} % use $\set*{1, 2, 3}$ 
\DeclarePairedDelimiter\abs{\lvert}{\rvert}
\DeclarePairedDelimiter\norm{\lVert}{\rVert}
\DeclarePairedDelimiter\ceils{\lceil}{\rceil}
\DeclarePairedDelimiter\floor{\lfloor}{\rfloor}
\DeclarePairedDelimiter\angles{\langle}{\rangle}
\def\Oh{\ensuremath{\mathcal{O}}} % big O like $\Oh(n)$
\def\oh{\ensuremath{\scriptstyle{\mathcal{O}}}} % small O
% ---

\begin{document}

\begin{small}
    \noindent
    Schubert Julian, Gruppe 4 \\
    Philipp Wahl, Gruppe 4
\end{small}
\bigskip

\begin{center}
    \LARGE Abgabe zum 8. Übungsblatt (AGT 21)
\end{center}
\smallskip
\subsection*{Aufgabe 1:}
\paragraph*{a)}
Wir widerlegen die Behauptung unseres Kommikitonen mit einem Beispiel:
\begin{figure}[H]
    \centering
    \begin{tikzpicture}[>=stealth',shorten >=1pt,node distance=2cm,on grid,initial/.style,     ={}]
            \node[state]          (1)                  {1};
            \node[state]          (2) [right =of 1]    {2};

            % Left cluster
            \node[state]          (3) [above left =of 1]    {3};
            \node[state]          (4) [below left =of 1]    {4};
            \node[state]          (5) [above left =of 4]    {5};

            % Right cluster
            \node[state]          (6) [above right =of 2]    {6};
            \node[state]          (7) [below right =of 2]    {7};
            \node[state]          (8) [above right =of 7]    {8};

            \tikzset{mystyle/.style={-,double=black}}   
            \path (1)     edge [mystyle]  node  {} (2);

            % Left cluster
            \path (1)     edge [mystyle]  node  {} (3);
            \path (1)     edge [mystyle]  node  {} (4);
            \path (1)     edge [mystyle]  node  {} (5);
            \path (3)     edge [mystyle]  node  {} (4);
            \path (5)     edge [mystyle]  node  {} (3);
            \path (5)     edge [mystyle]  node  {} (4);

            % Right cluster
            \path (2)     edge [mystyle]  node  {} (6);
            \path (2)     edge [mystyle]  node  {} (7);
            \path (2)     edge [mystyle]  node  {} (8);
            \path (6)     edge [mystyle]  node  {} (7);
            \path (8)     edge [mystyle]  node  {} (6);
            \path (8)     edge [mystyle]  node  {} (7);

    \end{tikzpicture}
\caption{Graph G=(V, E)}
\end{figure}
Der Ansatzt von unserem Kommilitonen würde in diesem Beispiel als Schnitt beispielsweise $(\{4\}, V \backslash \{4\})$ oder
$(\{7\}, V \backslash \{7\})$ liefern (da dies die beiden Knoten mit höchstem Grad sind). In diesem
Beispiel gibt es jedoch einen kleineren (minimalen) möglichen Schnitt: $(\{1, 3, 4, 5\}, \{2, 6, 8, 7\})$ \\
Damit ist die Behauptung des Kommilitonen widerlegt.
\paragraph*{b)}
Die WSK das Contract einen festen (kleinsten) Schnitt findet ist $\geq \frac{2}{n (n - 1)}$. Wir nehmen nun an, dass 
für $n \geq 2$ $V^2$ viele kleinste Schnitte existieren (und widerlegen die Aussage so per Kontraposition): 
\[
    \frac{2}{n(n - 1)} \cdot n^2 = \frac{2n^2}{n(n - 1)}  
\]
Da hier der Zähler des Bruchs für $n \in \mathbb{N}$ größer als der Nenner des Bruchs ist, würde sich dafür,
dass Contract einen der kleinsten Schnitte von größer als 1 ergeben. Damit muss gelten, dass es höchstens
\Oh($|V|^2$) viele verschiedenen kleinste Schnitte geben kann.
\subsection*{Aufgabe 2:}
Zunächst zeigen wir, wie man eine Zufällige Kante in \Oh(V) wählen kann: \\
\noindent GetRandomEdge($G = (V, E)$)\\
\begin{algorithm}[H]
    total = $\sum_{v \in V} deg(v)$ \\
    rdm = Zufallszahl zwischen 1 und total (beides inklusive) \\
	sum = 0 \;
    \For(){$v \in V$}{
        \If(Kante zurück geben){sum + v.deg() $\geq$ rdm}{
            \Return v.adj[rdm - (sum + v.deg())]
        } \Else(sum um den Grad vom Knoten v erhöhen){
            sum = sum + v.deg()
        }
    }

	\Return nil\;
\end{algorithm}
Dieser Algorithmus für das Zufällige auswählen einer Kante dauert \Oh(V) da wir zwei mal über alle Knoten 
iterieren (einmal um total zu bestimmen und einmal für die Kante). \\
Nun müssen wir nur noch den Graphen fertig Kontrahieren: \\
Unsere neue Kante verbindet die Knoten $v_i$ und $v_j$. Nun iterieren wir über alle Einträge
in der Adjazenzliste von $v_i$ und ändern für jeden Knoten $v_k$ in der Adjazenzliste von $v_i$
den Eintrag $v_i$ in $v_j$ um. Wir ersetzen also alle Kanten von $v_k$ nach $v_i$ zu Kanten die 
von (bzw zu, selbes Prinzip) $v_k$ zu $v_j$ gehen. \\
Wir wählen weiterhin Kanten gleichverteilt aus da jede Kante mit unserem Algorithmus die selbe chance
hat ausgewählt zu werden (da Knoten mit höhterem Grad nun eine höhere Chance haben ausgewählt zu werden)
\subsection*{Aufgabe 3:}
\paragraph*{a)}
Wir widerlegen wieder durch ein Gegenbeispiel:
\begin{figure}[H]
    \centering
    \begin{tikzpicture}[>=stealth',shorten >=1pt,node distance=2cm,on grid,initial/.style,     ={}]
            \node[state]          (1)                  {1};
            \node[state]          (2) [below =of 1]    {2};
            \node[state]          (3) [below =of 2]    {3};
            \node[state]          (4) [below =of 3]    {4};
            \node[state]          (5) [below =of 4]    {5};
            \node[state]          (6) [below =of 5]    {6};
            \node[state]          (7) [below =of 6]    {7};
            \node[state]          (8) [below =of 7]    {8};

            \node[state]          (9) [right =of 1]    {9};
            \node[state]          (10) [right =of 3]    {10};
            \node[state]          (11) [right =of 5]    {11};
            \node[state]          (12) [right =of 7]    {12};

            \node[state]          (13) [right =of 9]    {13};


            \tikzset{mystyle/.style={-,double=black}}   
            \path (1)     edge [mystyle]  node  {} (9);
            \path (2)     edge [mystyle]  node  {} (9);
            \path (3)     edge [mystyle]  node  {} (10);
            \path (4)     edge [mystyle]  node  {} (10);
            \path (5)     edge [mystyle]  node  {} (11);
            \path (6)     edge [mystyle]  node  {} (11);
            \path (7)     edge [mystyle]  node  {} (12);
            \path (8)     edge [mystyle]  node  {} (12);

            \path (9)     edge [mystyle]  node  {} (13);
            \path (10)     edge [mystyle]  node  {} (13);
            \path (11)     edge [mystyle]  node  {} (13);
            \path (12)     edge [mystyle]  node  {} (13);



    \end{tikzpicture}
\caption{Graph G=(V, E)}
\end{figure}
Der Vorschlag unseres Kommilitonen würde den Schnitt  $(\{13\}, \{V \backslash \{13\}\})$ liefern, größer
ist jedoch Beispielsweise der Schnitt  $(\{1, 2, 3, 4, 5, 6, 7, 8\}, \{9 , 10, 11, 12, 13\})$
\paragraph*{b)}
Wenn eine Kante e auf dem Schnitt liegt, dürfen ihre beiden Endpunkte nicht beide in S bzw. nicht
beide in T liegen. Ist der eine Endpunkt nun in S (oder T), so ist die Wahrscheinlicheit dass der 
andere Endpunkt in T (bzw. S) liegt gleich 0.5. \\
$\Rightarrow$ die Wahrscheinlicheit liegt bei 0.5
\paragraph*{c)}
Für einen festen Schnitt ist für jeden Knoten vorgegeben, ob er in S oder in T liegen muss. 
Bedeutet jeder Knoten liegt mit einer WSK von 0.5 im richtigen Schnitt. Dadurch
ergibt sich 
\[
    P(E) = 0.5^{|V|}  
\]
Da jedoch wenn ALLE Knoten in der Falschen Menge liegen ebenfalls der Korrekte Schnitt
geliefert wird (S, T) = (T, S), gilt insgesamt:
\[
    P(E) = (0.5^{|V|}) \cdot 2
\]
\paragraph*{d)}
Wir zeigen zunächst das die Erwartete Anzahl an Kanten die auf dem Schnitt liegen, exakt
$|E| / 2$ ist: \\
Es existieren $|E|$ viele Kanten, für jede dieser Kanten ist laut Teilaufgabe b die WSK
das sie auf dem Schnitt liegt gleich 0.5. Damit liegen erwartet $|E| * 0.5 = |E| / 2$ viele
Kanten auf dem Schnitt. \\
Eine maximale Optimale Lösung ist die, in der ALLE Kanten auf dem Schnitt liegen. Das ist
$|E|$, also doppelt so groß wie die Anzahl an Kanten die unser Algorithmus zurück gibt.
Damit ergibt sich eine Güte von $\frac{1}{2}$
\end{document}