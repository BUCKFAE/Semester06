\documentclass[14pt]{article}

% Packages
\usepackage[utf8]{inputenc}
\usepackage[german]{babel}
\usepackage{amssymb}
\usepackage{fancyhdr}
\usepackage[framemethod=TikZ]{mdframed}
\usepackage{amsthm}
\usepackage[T1]{fontenc}

\title{Algorithmische Graphentheorie}
\author{Julian Schubert}

% Definition
\mdfdefinestyle{theoremstyle}{
    linecolor=blue!50,
    linewidth=2pt,
    frametitlerule=true,
    frametitlebackgroundcolor=gray!20,
    innertopmargin=\topskip
}
\mdtheorem[style=theoremstyle]{definition}{Definition}

% Eigenschaft
\mdfdefinestyle{eigenschaftstyle}{
    linecolor=red!50,
    linewidth=2pt,
    frametitlerule=true,
    frametitlebackgroundcolor=gray!20,
    innertopmargin=\topskip
}
\mdtheorem[style=eigenschaftstyle]{eigenschaft}{Eigenschaft}


% Kopf- / Fußzeile
\makeatletter
\let\runauthor\@author
\let\runtitle\@title
\pagestyle{fancy}
\fancyhf{}
\rhead{\runtitle}
\lhead{\runauthor}
\cfoot{\thepage}

\newcommand{\subsubsubsection}[1]{\paragraph{#1}\mbox{}\\}
\setcounter{secnumdepth}{4}
\setcounter{tocdepth}{4}


\begin{document}


    \maketitle
    \tableofcontents

    \newpage
    \section{Wichtige Begriffe}
    \begin{definition}
        Ein gerichteter Graph $G$ ist \textbf{schwach} zusammenhängend
        wenn der darunterliegende ungerichtete Graph zusammenhängend 
        ist \\
        Ein gerichteter Graph $G$ ist \textbf{stark} zusammenhängend
        wenn es für jedes Knotenpaar $(u, v)$ einen gerichteten Weg 
        von $u$ nach $v$ gibt
    \end{definition}

    \section{Eulerkreise}
    \begin{definition}[Eulerkreis]
        Sei $G$ ein (un-)gerichteter Grpah. \\
        Ein Eulerkreis (-weg) in 
        $G$ ist ein Kreis (Weg), der jede \textbf{Kante} genau 
        einmal durchläuft. \\
        Ein Graph heißt \textbf{eulersch}, falls er einen 
        Eulerkreis enthält
    \end{definition}
    Ein Graph der nur einen Eulerweg aber keinen Eulerkreis 
    enthält, ist nicht eulersch!
    \begin{eigenschaft}[Satz von Euler]
        Sei $G$ ein ungerichteter und zsh. Graph.  \\
        Dann gilt:
        $G$ eulersch $\Leftrightarrow$ alle Knoten haben geraden Grad
    \end{eigenschaft}
    Bei gerichteten Graphen: indeg($v$) = outdeg($v$)
    \subsection{Eulerkreis finden}
        Man kann in $O(E)$ testen on G eulersch ist (Knotengrade zählen) \\
        Eulerkreis finden: \\
        Verwalte in jedem Knoten $v$ eien zeiger curr[$v$], der auf 
        den ersten unbenutzten Nachbarn $w$ zeigt
    
    \section{Hamiltonkreise}
    \begin{definition}[Hamiltonkreis NP-schwer]
        Sei $G$ ein (un-)gerichteter Graph. Ein Hamiltonkreis (-weg)
        in $G$ ist ein Kreis (Weg), der jeden \textbf{Knoten} genau
        einmal durchläuft.
    \end{definition}
    \begin{eigenschaft}[Satz von Bondy und Chvátal]
        Sei $G = (V, E)$ ein ungerichteter Graph mit $|V| \geq 3$ \\
        Seien $u$ und $v$ nicht-adjazente Knoen von G mit deg($u$)
        + deg($v$) $\geq n$ := |V|. Dann gilt: \\
        \indent G hamiltons $\Leftrightarrow$ G + $uv$ hamitlonsch 
    \end{eigenschaft}
    \begin{eigenschaft}[Satz von Dirac]
        Sei G = (V, E) ein ungerichteter Graph mit |V| $\geq$ 3.
        Falls jeder Knoten von G Grad $\geq$ |V| / 2 hat, so ist 
        G hamiltonsch
    \end{eigenschaft}
    TODO: Beweisen

    \section{Handlungsreisen (TSP)}
    Lösbar mit Algorithmmus von Bellman \& Held-Karp
\end{document} 