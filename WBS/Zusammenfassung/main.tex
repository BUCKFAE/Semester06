\documentclass[14pt]{article}

% Packages
\usepackage[utf8]{inputenc}
\usepackage[german]{babel}
\usepackage{amssymb}
\usepackage{fancyhdr}
\usepackage[framemethod=TikZ]{mdframed}
\usepackage{amsthm}
\usepackage{amsmath}
\usepackage[T1]{fontenc}
\usepackage{mathabx}

\title{Wissensbasierte Systeme}
\author{Julian Schubert}

% Definition
\mdfdefinestyle{theoremstyle}{
    linecolor=blue!50,
    linewidth=2pt,
    frametitlerule=true,
    frametitlebackgroundcolor=gray!20,
    innertopmargin=\topskip
}
\mdtheorem[style=theoremstyle]{definition}{Definition}

% Eigenschaft
\mdfdefinestyle{eigenschaftstyle}{
    linecolor=red!50,
    linewidth=2pt,
    frametitlerule=true,
    frametitlebackgroundcolor=gray!20,
    innertopmargin=\topskip
}
\mdtheorem[style=eigenschaftstyle]{eigenschaft}{Eigenschaft}


% Kopf- / Fußzeile
\makeatletter
\let\runauthor\@author
\let\runtitle\@title
\pagestyle{fancy}
\fancyhf{}
\rhead{\runtitle}
\lhead{\runauthor}
\cfoot{\thepage}

\newcommand{\subsubsubsection}[1]{\paragraph{#1}\mbox{}\\}
\setcounter{secnumdepth}{4}
\setcounter{tocdepth}{4}


\begin{document}


    \maketitle
    \tableofcontents

    \newpage

    \section{Einführung}
    \subsection{Was sind Wissensbasierte Systeme?}
        \begin{itemize}
            \item \textbf{Ziel}
            \begin{itemize}
                \item Lösen eines Problems durch Wissen und Inferenz
            \end{itemize}
            \item \textbf{Unterschied zu Neuronalen Netzen}
            \begin{itemize}
                \item Lösung erklärbar und kritisierbar
                \item Aufwändiger Wissenserwerbsprozess
            \end{itemize}
            \item \textbf{Arten von Wissen}
            \begin{itemize}
                \item Fakten, Wahrscheinlichkeiten
                \item Relationen, REgeln, Constraints
                \item Muster, Fälle + Ähnlichkeitsmaß
            \end{itemize}
            \item \textbf{Wissenserwerb}
            \begin{itemize}
                \item durch Fachexperten
                \item durch Lernen aus Fällen
                \item durch Extraktion aus Literatur
            \end{itemize}
        \end{itemize}

    \subsection{Zentrale Aufgaben}

    \textbf{Wissensrepräsentation festlegen} \\
    \begin{itemize}
        \item Basiert häufig auf einer Befragung von Fachexperten
        \begin{itemize}
            \item Für Fachexperten natürlich
            \item Präzise zur Herleitung von Schlussfolgerungen
            \item Effizient verarbeitbar
        \end{itemize}
    \end{itemize}
    \textbf{Wissen aquirieren} \\
    Editor zur Eingabe von Wissen wird benötigt
    \begin{itemize}
        \item Geringe Einarbeitungszeit
        \item Natürliche Darstellung
        \item Effiziente Wissenseingabe
        \item Übersichtlich auch für große Wissensbasen
        \item Sollte eine Schnittstelle zum Testen des Wissens bieten
    \end{itemize}
    \textbf{Wissen verarbeiten (Reasoning)} \\
    \textbf{Evaluation mit Fällen}

    \subsection{Interaktive und eingebettete WBS}
    \begin{itemize}
        \item \textbf{Interaktiv}
        \begin{itemize}
            \item WBS berät Nuzter in geführtem Dialog
            \item WBS präsentiert Lösung(en) mit vorheriger 
            Dateneingabe
            \item WBS unterstützt Exploration des Lösungsraums
        \end{itemize}
        \item \textbf{Embedded}
        \begin{itemize}
            \item WBS präsentiert Lösung ohne Dateneingabe
            \item WBS gibt Hinweise (Alerts), falls notwendig 
            (z.B. Kritik)
            \item WBS handelt autonom
        \end{itemize}
    \end{itemize}

    \subsection{Lebenszyklus eines WBS}
    \begin{itemize}
        \item \textbf{Bedarf feststellen:} Ist-Zustand, Ziele
        \item \textbf{Entwickeln:} Methoden, Phasen
        \item \textbf{Bereitstellen:} z.B. Server, Integration in 
        anderes System
        \item \textbf{Nutzen:} GUI, autonom
        \item \textbf{Evaluieren:} Korrektheit, Zeitersparnis,
        Dokumentation
        \item \textbf{Evolvieren:} Lernen, Weiterentwickeln
    \end{itemize}

    \subsection{Domain Specific Languages (DSL)}
    Abgrenzung zu WBS, trotz ähnlicher Zielsetzung: Formale Sprache 
    Programmierung und Wissenformalisierung in einer eingeschränkten
    Domäne für Domänenspezialisten \\
    \subsubsection{Interne DSL}
    \textbf{Untermenge einer generellen Sprache}, z.B. UML-Profile, 
    domänenspezifische XML-Schemata
    \subsubsection{Externe DSL}
    \textbf{Neu definiert}, z.B. SQL, reguläre Ausdrücke

    \subsection{Wissenserwerb}
    \subsubsection{Heuristische Entscheidungsbäume}
        Man benutzt nicht einen großen Entscheidungsbaum (ein mal 
        falsch abbiegen und man kommt eventuell nicht mehr auf
        die richtige Lösung), sondern mehrere kleine Bäume
    \subsubsection{Überdeckender Diagnose-Score}
        Tabelle mit Baum und Merkmale, Einträge: Wie 
        Wahrscheinlich ist das Attribut für den Baum
    \subsubsection{Modellbasierter Wissenserwerb}
        Modell erstellen und daran lösung erarbeiten
    \subsubsection{Fallorentierter Wissenserwerb}
\end{document}