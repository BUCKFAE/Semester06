\documentclass[14pt]{article}

% Packages
\usepackage[utf8]{inputenc}
\usepackage[german]{babel}
\usepackage{amssymb}
\usepackage{fancyhdr}
\usepackage[framemethod=TikZ]{mdframed}
\usepackage{amsthm}
\usepackage{amsmath}
\usepackage[T1]{fontenc}
\usepackage{mathabx}

\title{DataMining CheatSheet}
\author{Julian Schubert}

% Definition
\mdfdefinestyle{theoremstyle}{
    linecolor=blue!20,
    linewidth=2pt,
    frametitlerule=true,
    frametitlebackgroundcolor=gray!20,
    innertopmargin=\topskip
}
\mdtheorem[style=theoremstyle]{definition}{Definition}

% Eigenschaft
\mdfdefinestyle{eigenschaftstyle}{
    linecolor=red!50,
    linewidth=2pt,
    frametitlerule=true,
    frametitlebackgroundcolor=gray!20,
    innertopmargin=\topskip
}
\mdtheorem[style=eigenschaftstyle]{eigenschaft}{Eigenschaft}


% Kopf- / Fußzeile
\makeatletter
\let\runauthor\@author
\let\runtitle\@title
\pagestyle{fancy}
\fancyhf{}
\rhead{\runtitle}
\lhead{\runauthor}
\cfoot{\thepage}

\newcommand{\subsubsubsection}[1]{\paragraph{#1}\mbox{}\\}
\setcounter{secnumdepth}{4}
\setcounter{tocdepth}{4}

\begin{document}

\maketitle

\section{Gütemaße}
\subsection{Davies-Bouldin Index (DB)}
    \begin{tabular}{| c | c |}
        \hline
        Güte innerhalb des Clusters $C_i$ 
            & $S_i \sqrt[q]{\frac{1}{|C_i|} \sum_{x \in C_i}\text{dist}(x, \mu_i)^q}$ \\
        \hline
        Güte der Trennung der Cluster $C_i$ und $C_j$
            & $M_{i,j} = $dist$(\mu_i, \mu_j)$ \\
        \hline
        $R_{i,j}$ für $i \neq j$
            & $R_{i, j} = \frac{S_i + S_j}{M_{i, j}}$ \\
        \hline
        Davis-Bouldin Index
            & $DB = \frac{1}{k} \sum_{i = 1}^k D_i$ mit $D_i = \max_{i \neq j} R_{i,j}$ \\
        \hline
    \end{tabular}
\end{document}