\documentclass[14pt]{article}

% Packages
\usepackage[utf8]{inputenc}
\usepackage[german]{babel}
\usepackage{amssymb}
\usepackage{fancyhdr}
\usepackage[framemethod=TikZ]{mdframed}
\usepackage{amsthm}
\usepackage{amsmath}
\usepackage[T1]{fontenc}
\usepackage{mathabx}

\title{Statistik Basics}
\author{Julian Schubert}

% Definition
\mdfdefinestyle{theoremstyle}{
    linecolor=blue!50,
    linewidth=2pt,
    frametitlerule=true,
    frametitlebackgroundcolor=gray!20,
    innertopmargin=\topskip
}
\mdtheorem[style=theoremstyle]{definition}{Definition}

% Eigenschaft
\mdfdefinestyle{eigenschaftstyle}{
    linecolor=red!50,
    linewidth=2pt,
    frametitlerule=true,
    frametitlebackgroundcolor=gray!20,
    innertopmargin=\topskip
}
\mdtheorem[style=eigenschaftstyle]{eigenschaft}{Eigenschaft}


% Kopf- / Fußzeile
\makeatletter
\let\runauthor\@author
\let\runtitle\@title
\pagestyle{fancy}
\fancyhf{}
\rhead{\runtitle}
\lhead{\runauthor}
\cfoot{\thepage}

\newcommand{\subsubsubsection}[1]{\paragraph{#1}\mbox{}\\}
\setcounter{secnumdepth}{4}
\setcounter{tocdepth}{4}


\begin{document}

\section{Standardabweichung}
    \begin{definition}[Standardabweichung]
        Die \textbf{Standardabweichung} gibt die Streubreite einer Variable rund 
        um deren Mittelwert an. Damit ist die Standardabweichung die 
        durchschnittliche Entfernung aller gemessenen Werte einer Variable vom 
        Mittelwert der Verteilung.
    \end{definition}
    \subsection{Berechnung}
    Die Standardabweichung $\sigma$ lässt sich wie folgt berechnen: 
    \[
        \sigma = \sqrt{\frac{1}{n} \sum_{i=0}^n(x_i - \text{Mittelwert})^2}
    \]

\section{Varianz}
    \begin{definition}[Varianz]
        Die \textbf{Varianz} ist das Quadrat der Standardabweichung
    \end{definition}

\section{Relative Häufigkeit / Wahrscheinlichkeit}
    Die \textbf{relative Häufigkeit} lässt sich wie folgt berechnen:
    \[
        \text{relative Häufigkeit} 
            = \frac{\text{Ergebnisanzahl}}{\text{Versuchsanzahl}}  
    \]
    Daraus lässt sich dann die \textbf{relative Wahrscheinlichkeit}
    berechne.

\section{Wahrscheinlichkeiten in Abhängigkeit}
    Es gilt: 
    \[
        P(A | B) = \frac{P(A \cap B)}{P(B)}
    \]
    \textbf{Satz von Bayes:}
    \[
        P(A | B) = \frac{P(B | A) \cdot P(A)}{P(B)}
    \]

\section{Binominalkoeffizient}
    \begin{definition}[Binominalkoeffizient]
        Der \textbf{Binomialkoeffizient} $\binom{n}{k}$ gibt an auf 
        wie viele verschiedene Arten man  $k$ bestimmte Objekte aus 
        einer Menge von $n$ Objekten auswählen kann. \\
        Er ist also die Anzahl der $k$-elementigen Teilmengen 
        einer $n$-elementigen Menge.
    \end{definition}
    Es gilt:
    \[
        \binom{n}{k} = \frac{n!}{k! \cdot (n - k)!}
    \]  

\section{Binomialverteilung}
    \begin{definition}[Binomialverteilung]
        Ist $p$ die Erfolgswahrscheinlichkeit bei einem Versuch und $n$
        die Anzahl der Versuche, dann bezeichnet man mit $B(k | p, n)$
        die Wahrscheinlichkeit, genau $k$ Erfolge zu erzielen.
    \end{definition}
        
    \subsection{Berechnung}
        \[
            B(k | p, n) = \binom{n}{k} p^k (1 - p)^{n - k} 
        \]
\end{document}